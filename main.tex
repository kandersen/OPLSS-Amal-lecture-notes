% Dokumentklassen sættes til memoir.
% Manual: http://ctan.org/tex-archive/macros/latex/contrib/memoir/memman.pdf
\documentclass[a4paper,10pt,fleqn]{article}
\usepackage[a4paper]{geometry}
% Danske udtryk (fx figur og tabel) samt dansk orddeling og fonte med

% danske tegn. Hvis LaTeX brokker sig over æ, ø og å skal du udskifte
% "utf8" med "latin1" eller "applemac". 
\usepackage[utf8]{inputenc}
%\usepackage[danish]{babel}
\usepackage[T1]{fontenc}
\usepackage{fixltx2e} 
\usepackage{color}
\usepackage{hyperref}

% Matematisk udtryk, fede symboler, theoremer og fancy ting (fx kædebrøker)
\usepackage{amsmath,amssymb}
\usepackage{stmaryrd}
\usepackage{bm}
\usepackage{amsthm}
\usepackage{tikz}
\usetikzlibrary{arrows}
\usetikzlibrary{positioning}
\usetikzlibrary{snakes}
\usetikzlibrary{decorations.pathreplacing}
\usepackage{mathtools}
% Kodelisting. Husk at læse manualen hvis du vil lave fancy ting.
% Manual: http://mirror.ctan.org/macros/latex/contrib/listings/listings.pdf
\usepackage{listings}
\usepackage{verbatim}
\usepackage{enumitem}
\usepackage{bussproofs}
\usepackage{mathpartir}
 
% Fancy ting med enheder og datatabeller. Læs manualen til pakken
% Manual: http://www.ctan.org/tex-archive/macros/latex/contrib/siunitx/siunitx.pdf
%\usepackage{siunitx}
% Indsættelse af grafik.
\usepackage{graphicx}
\usepackage{xcolor}
%\usepackage{bussproofs}

% Reaktionsskemaer. Læs manualen for at se eksempler.
% Manual: http://www.ctan.org/tex-archive/macros/latex/contrib/mhchem/mhchem.pdf
%\usepackage[version=3]{mhchem}

\setlength{\mathindent{1cm}}

\newcommand{\sem}[1]{\ensuremath{\llbracket #1 \rrbracket}}
\newcommand{\curly}[1]{\ensuremath{\mathcal{#1}}}

\DeclareMathOperator{\FTV}{FTV}
\DeclareMathOperator{\dom}{dom}
\DeclareMathOperator{\safe}{safe}
\DeclareMathOperator{\irred}{irred}
\DeclareMathOperator{\SNPred}{SN}
\DeclareMathOperator{\val}{Val}
\DeclareMathOperator{\fst}{fst}
\DeclareMathOperator{\snd}{snd}
\DeclareMathOperator{\inl}{inl}
\DeclareMathOperator{\inr}{inr}
\DeclareMathOperator{\scase}{case}
\DeclareMathOperator{\caseof}{of}
\DeclareMathOperator{\fold}{fold}
\DeclareMathOperator{\unfold}{unfold}

\newcommand{\evalto}{\ensuremath{\mapsto}}
\newcommand{\evaltos}[1][*]{\ensuremath{\evalto^{#1}}}
\newcommand{\warning}[1]{{\color{red} !` #1 !} \\}
\newcommand{\mtenv}{\ensuremath{\bullet}}
\newcommand{\case}[1]{\\{\bf Case} #1,}
\newcommand{\eqdef}{\ensuremath{ \; \stackrel{\mathclap{\normalfont\mbox{def}}}{=} \;}}
\newcommand{\subst}[3]{\ensuremath{\ensuremath{#1[#2/#3]}}}
\newcommand{\labs}[2]{\ensuremath{\lambda #1 . \; #2}}
\newcommand{\tlabs}[3]{\ensuremath{\lambda #1 : #2 . \; #3}}
\newcommand{\tLabs}[2][\alpha]{\ensuremath{\Lambda #1 . \; #2}}
\newcommand{\vbar}{\ensuremath{\; | \;}}
\newcommand{\tarrow}[2]{\ensuremath{ #1 \rightarrow #2}}
\newcommand{\eif}[3]{\ensuremath{ \text{if}\; #1 \; \text{then} \; #2 \; \text{else} \; #3}}
\newcommand{\true}{\ensuremath{\text{true}}}
\newcommand{\false}{\ensuremath{\text{false}}}
\newcommand{\SN}[2]{\ensuremath{\SNPred_{#1}(#2)}}

\newcommand{\pred}[2]{\ensuremath{\curly{#1}\sem{#2}}}
\newcommand{\pres}[3]{\ensuremath{\curly{#1}_{#2}\sem{#3}}}

\newcommand{\epred}[1]{\ensuremath{\pred{E}{#1}}}
\newcommand{\epres}[2][k]{\ensuremath{\pres{E}{#1}{#2}}}

\newcommand{\vpred}[1]{\ensuremath{\pred{V}{#1}}}
\newcommand{\vpres}[2][k]{\ensuremath{\pres{V}{#1}{#2}}}

\newcommand{\gpred}[1]{\ensuremath{\pred{G}{#1}}}
\newcommand{\gpres}[2][k]{\ensuremath{\pres{G}{#1}{#2}}}

\newcommand{\sub}[3]{\ensuremath{#1[^{#2}/_{#3}]}}

\newcommand*{\circled}[1]{\tikz[baseline=(char.base)]{
            \node[shape=circle,draw,inner sep=2pt] (char) {#1};}}
\newcommand{\TTrue}{\ensuremath{
    \inferrule*[right=T-True]{ }
               {\Gamma \vdash \true : bool}}}

\newcommand{\TFalse}{\ensuremath{
    \inferrule*[right=T-False]{ }
               {\Gamma \vdash \false : bool}}}


\newcommand{\TVar}{\ensuremath{
    \inferrule*[right=T-Var]{\Gamma(x) = \tau}
                            {\Gamma \vdash x : \tau}}}
\newcommand{\TIf}{\ensuremath{
    \inferrule*[right=T-If]{\Gamma \vdash e : bool \and \Gamma \vdash e_1 : \tau \and \Gamma \vdash e_2 : \tau}
               {\Gamma \vdash \eif{e}{e_1}{e_2} : \tau}}}
\newcommand{\TApp}{\ensuremath{
    \inferrule*[right=T-App]{\Gamma \vdash e_1 : \tarrow{\tau_2}{\tau} \and
                            \Gamma \vdash e_2 : \tau_2}
                           {\Gamma \vdash e_1 \; e_2 : \tau}}}

\newcommand{\TAbs}{\ensuremath{\inferrule*[right=T-Abs]{\Gamma, x: \tau_1 \vdash e : \tau_2}
                           {\Gamma \vdash \tlabs{x}{\tau_1}{e} : \tarrow{\tau_1}{\tau_2}}}}
\newcommand{\TFold}{\ensuremath{
    \inferrule*[right=T-Fold]{\Gamma \vdash e : \tau[\mu\alpha. \; \tau/\alpha]}
                             {\Gamma \vdash \fold \; e : \mu\alpha. \; \tau}}}
\newcommand{\TUnfold}{\ensuremath{
    \inferrule*[right=T-Unfold]{\Gamma \vdash e : \mu\alpha. \; \tau}
                               {\Gamma \vdash \unfold \; e : \tau[\mu\alpha. \; \tau/\alpha]}}}


\newcommand{\FTTrue}{\ensuremath{
    \inferrule*[right=T-True]{ \Delta \vdash \Gamma}
               {\Gamma \vdash \true : bool}}}

\newcommand{\FTFalse}{\ensuremath{
    \inferrule*[right=T-False]{\Delta \vdash \Gamma }
               {\Gamma \vdash \false : bool}}}

\newcommand{\FTVar}{\ensuremath{
    \inferrule*[right=T-Var]{\Gamma(x) = \tau \and \Delta \vdash \Gamma}
                            {\Gamma \vdash x : \tau}}}
\newcommand{\FTApp}{\ensuremath{
    \inferrule*[right=T-App]{\Gamma \vdash e_1 : \tarrow{\tau_2}{\tau} \and
                            \Gamma \vdash e_2 : \tau_2 \and 
                            \Delta \vdash \Gamma}
                           {\Gamma \vdash e_1 \; e_2 : \tau}}}

\newcommand{\FTAbs}{\ensuremath{\inferrule*[right=T-Abs]{\Gamma, x: \tau_1 \vdash e : \tau_2 \and \Delta \vdash \Gamma}
                           {\Gamma \vdash \tlabs{x}{\tau_1}{e} : \tarrow{\tau_1}{\tau_2}}}}
\newcommand{\FTIf}{\ensuremath{
    \inferrule*[right=T-If]{\Gamma \vdash e : bool \and 
                            \Gamma \vdash e_1 : \tau \\ 
                            \Gamma \vdash e_2 : \tau \and 
                            \Delta \vdash \Gamma}
               {\Gamma \vdash \eif{e}{e_1}{e_2} : \tau}}}



\newtheorem*{theorem}{Theorem}
\newtheorem*{lemma}{Lemma}
%Lecture 1:
\newtheorem*{strnorm}{Theorem}
\newtheorem*{astrnorm}{Theorem}
\newtheorem*{substlem}{Lemma}
\newtheorem*{forback}{Lemma}
%Lecture 2:
\newtheorem*{stlctypesafety}{Theorem}
\newtheorem*{progress}{Lemma}
\newtheorem*{preservation}{Lemma}
\newtheorem*{btypesafety}{Theorem}
%Lecture 3:
\newtheorem*{stlcmutypesafety}{Theorem}
\newtheorem*{stlcmufundprop}{Theorem}
\newtheorem*{monotonicity}{Lemma}

\author{Lau\\lask@cs.au.dk}
\title{OPLSS15, Notes for Amal Ahmed's lectures.}
\begin{document}
\maketitle
The videos of the lectures of OPLSS 2015 can be found at \url{https://www.cs.uoregon.edu/research/summerschool/summer15/curriculum.html}
\section*{Lecture 1, Logical Predicates and Relations}
\subsection*{Simply Typed Lambda Calculus (STLC)}
The language used to present logical predicates and relations is the simply typed lambda calculus. In the first couple lectures it will be used as it is presented here. In the later sections simply typed lambda calculus will be used as a base language so if it later on says that we extend with some construct, then it is the simply typed lambda calculus that we are extending. The simply typed lambda calculus is defined as follows:\\
\begin{tabular}{ r | l }
  Types: & $\tau ::=  bool \vbar \tarrow{\tau}{\tau}$ \\
  \hline
  Terms: & $e    ::= x \vbar \true 
                       \vbar \false       
                       \vbar \eif{e}{e}{e} 
                       \vbar \tlabs{x}{\tau }{e}
                       \vbar e \; e$ \\
  \hline
  Values: & $v    ::= \true \vbar \false \vbar \tlabs{x}{\tau}{e}$ \\
  \hline
  Evaluation  & \multirow{2}{*}{$E    ::= [] \vbar \eif{E}{e}{e} \vbar E \; e \vbar v \; E$}\\
  contexts: & \\
  \hline
  \multirow{4}{*}{Evaluations:}    
                                   & $\eif{\true}{e_1}{e_2} \evalto e_1$ \\
                                   & $\eif{\false}{e_1}{e_2} \evalto e_2$ \\
                                   & $(\tlabs{x}{\tau}{e}) \; v \evalto \subst{e}{v}{x}$ \\
                                   & $\inferrule*[]{e \evalto e'}
                                                   {E[e] \evalto E[e']}$ \\
  \hline
  Typing & \multirow{2}{*}{$ \Gamma ::= \mtenv \vbar \Gamma , x : \tau$} \\
  Contexts: & \\
  \hline
  \multirow{8}{*}{Typing rules:} & \\
                                 & $\TFalse \hspace{1.2cm} \TTrue$ \\
                                 & \\
                                 & $\TVar \hspace{1.2cm} \TIf$ \\
                                 & \\
                                 & $\TAbs \hspace{1.2cm} \TApp$\\
                                 & \\
\end{tabular}\\
For the typing contexts it is assumed that the binders are distinct. So if $x \in \dom(\Gamma)$, then $\Gamma , x : \tau$ is not a legal context.
\subsection*{Logical Relations}
Logical relations are used to prove properties about programs in a language. Logical relations are proof methods and can be used as an alternative to proving properties directly. Examples of properties one can show using logical relations are:
\begin{itemize}
\item Termination (Strong normalization)
\item Type safety
\item Equivalence of programs
  \begin{itemize}
  \item Correctness of programs
  \item Representation independence
  \item Parametricity and free theorems, e.g.,
    \[
    f: \forall \alpha. \; \tarrow{\alpha}{\alpha}
    \]
    The program cannot inspect $\alpha$ as it has no idea which type it will be, therefore $f$ must be identity function.
    \[
    \forall. \; \tarrow{int}{\alpha}
    \]
    A function with this type does not exist (the function would need to return something of type $\alpha$, but it only has something of type $int$ to work with so it cannot possibly return a value of the proper type).
  \item Security-Typed Languages (for Information Flow Control (IFC))\\
        Example: All types in the code snippet below are labeled with their security level. A type can be labeled with either $L$ for \emph{low} or $H$ for \emph{high}. We do not want any flow from variables with a \emph{high} labeled type to a variable with a \emph{low} labeled type. The following is an example of an insecure \emph{explicit flow} of information:
        \begin{lstlisting}[escapeinside={@}{@}]
  x : int@$^L$@
  y : int@$^H$@
  x = y    //This assignment is insecure.
        \end{lstlisting}
Further, information may leak through a \emph{side channel}. That is the value denoted by a variable with a \emph{low} labeled type depends on the value of a variable with a \emph{high} labeled type. If this is the case we may not have learned the secret value, but we may have learned some information about it. An example of a side channel:
        \begin{lstlisting}[escapeinside={@}{@}]
  x : int@$^L$@
  y : int@$^H$@
  if y > 0 then x = 0 else x = 1
        \end{lstlisting}
The above examples show undesired programs or parts of programs, but if we want to generally state behavior we do not want a program to show, then we state it as non-interference:
\begin{align*}
  & \vdash P : \tarrow{int^L \times int^H}{int^L} \\
  & P(v_L,v_{1H}) \approx_L P(v_L,v_{2H})
\end{align*}
If we run $P$ with the same \emph{low} value and with two different \emph{high} values, then the \emph{low} result of the two runs of the program should be equal. That is the \emph{low} result does not depend on \emph{high} values.
  \end{itemize}
\end{itemize}
\subsection*{Categories of Logical Relations}
We can split logical relations into two logical relations and logical predicates. Logical predicates are unary and used usually used to show properties of a program. Logical relations are binary and are usually used to show equivalences:
\begin{tabular}{l | l}
  Logical Predicates     & Logical Relations    \\
\hline
  (Unary)                & (Binary)             \\
  $P_\tau(e)$             & $R_\tau(e_1,e_2)$     \\
  - One property         & - Program Equivalence\\ %\footnote{Was not in my notes.}
  - Strong normalization & \\
  - Type safety          & \\
\end{tabular}
\subsection*{Strong Normalization of STLC}
In this section we wish to show that the simply typed lambda calculus has strong normalization which means that every term is strongly normalizing. Normalization of a term the process of reducing a term into its normal form. If a term is strongly normalizing, then it reduces to its normal form. In our case we define the normal forms of the language to be the values of the language.
\subsubsection*{A first try on normalization of STLC}
We start with a couple of abbreviations:
\begin{align*}
  e \Downarrow v & \eqdef e \evaltos v \\
  e \Downarrow   & \eqdef \exists v. \; e \Downarrow v
\end{align*}
Where $v$ is a value.
What we want to prove is:
\begin{strnorm}[Strong Normalization]~\\
  If $\mtenv \vdash e : \tau$ then $e \Downarrow$
\end{strnorm}
\begin{proof} 
\warning{This proof gets stuck and is not complete. It is included to motivate the use of a logical predicate.}
Induction on the structure of the typing derivation.
\case{$\mtenv \vdash \true : bool$} this term has already terminated.
\case{$\mtenv \vdash \false : bool$} same as for \true.
\case{$\mtenv \vdash \eif{e}{e_1}{e_2} : \tau$} simple, but requires the use of canonical forms of bool.
\case{$\mtenv \vdash \tlabs{x}{\tau_1}{e}$ : \tarrow{\tau_1}{\tau_2}} it is a value already and it has terminated.
\case{$ \TApp $} \\
By the induction hypothesis we get $e_1 \Downarrow v_1$ and $e_2 \Downarrow v_2$. By the type of $e_1$ we conclude $e_1 \Downarrow \tlabs{x}{\tau_2}{e'}$. What we need to show is $e_1 \; e_2 \Downarrow$. We know $e_1 \; e_2$ takes the following steps:
\begin{align*}
  e_1 \; e_2 & \evaltos (\tlabs{x}{\tau_2}{e'}) \; e_2 \\
            & \evaltos (\tlabs{x}{\tau_2}{e'}) \; v_2 \\
            & \evalto e'[v_2/x]
\end{align*}
Here we run into an issue as we do not know anything about $e'$. Our induction hypothesis is not strong enough.\footnote{:(}
\end{proof}
\subsubsection*{A logical predicate for strongly normalizing expressions}
We want to define a logical predicate, \SN{\tau}{e}. We want $\SNPred_\tau$ to accept the expressions of type $\tau$ that are strongly normalizing, but let us first consider what properties we in general want a logical predicate to have.

In general for a logical predicate, $P_\tau(e)$, we want an expression, $e$, accepted by this predicate to satisfy the following properties\footnote{Note: when we later want to prove type safety we do not want well-typedness to be a property of the predicate.}:
\begin{enumerate}
\item $\mtenv \vdash e : \tau$
\item The property we wish $e$ to have. In this case it would be: $e$ is strongly normalizing.
\item The condition is preserved by eliminating forms.
\end{enumerate}
With the above in mind we define the strongly normalizing predicate as follows:
\begin{align*}
  \SN{bool}{e} & \Leftrightarrow \mtenv \vdash e : bool \pand e \Downarrow \\
  \SN{\tarrow{\tau_1}{\tau_2}}{e} & \Leftrightarrow \mtenv \vdash e : \tarrow{\tau_1}{\tau_2} \pand e \Downarrow \pand (\forall e'. \; \SN{\tau_1}{e'} \implies \SN{\tau_2}{e \; e'})\\
\end{align*}
It is here important to consider whether the logical predicate is well-founded. \SN{\tau}{e} is defined over the structure of $\tau$, so it is indeed well-founded.
\subsubsection*{Strongly normalizing using a logical predicate}
We are now ready to show strong normalization using \SN{\tau}{e}. The proof is done in two steps:
\[
\circled{a} \quad \mtenv \vdash e : \tau \implies \SN{\tau}{e}
\]
\[
\circled{b} \quad \SN{\tau}{e} \implies e \Downarrow
\]
The structure of this proof is common to proofs that use logical relations. We first prove that well-typed terms are in the relation. Then we prove that terms in the relation actually have the property we want to show (in this case strong normalization).

The proof of \circled{b} is by induction on $\tau$.\footnote{This should not be difficult, as we baked the property we want into the relation. That was the second property we in general wanted a logical relation to satisfy.}

We could try to prove \circled{a} by induction over $\mtenv \vdash e : \tau$, but the case
\[
  \TAbs
\]
gives issues. Instead we prove a generalization of \circled{a}
\begin{astrnorm}[\circled{a} Generalized]
  If $\Gamma \vdash e : \tau$ and $\gamma \models \Gamma$ then $\SN{\tau}{\gamma(e)}$
\end{astrnorm}
Here $\gamma$ is a substitution, $\gamma = \{x_1 \mapsto v_1, \dots , x_n \mapsto v_n\}$. We define the substitution to work as follows:
\begin{align*}
  & \emptyset (e) = e \\
  & \extsub{\gamma}{x}{v}(e) = \gamma(\subst{e}{x}{v})
\end{align*}

In English the theorem reads: If $e$ is well-typed with respect to some type $\tau$ and we have some closing substitution that satisfy the typing environment, then if we close of $e$ with $\gamma$, then this closed expression is in $\SNPred_\tau$.

$\gamma \models \Gamma$ is read ``the substitution, $\gamma$, satisfies the type environment, $\Gamma$.'' It is defined as follows:
\[
  \gamma \models \Gamma \eqdef \dom(\gamma) = \dom(\Gamma) \pand 
                 \forall x \in \dom(\Gamma). \; \SN{\Gamma(x)}{\gamma(x)}
\]
To prove the generalized theorem we need further two lemmas
\begin{substlem}[Substitution lemma]
  If $\Gamma \vdash e : \tau$ and $\gamma \models \Gamma$ then $\mtenv \vdash \gamma (e) : \tau$
\end{substlem}
\begin{forback}[$\SNPred$ preserved by forward/backward reduction]
  Suppose $\mtenv \vdash e : \tau$ and $e \evalto e'$
  \begin{enumerate}
  \item if $\SN{\tau}{e'}$ then $\SN{\tau}{e}$
  \item if $\SN{\tau}{e}$ then $\SN{\tau}{e'}$
  \end{enumerate}
\end{forback}
\begin{proof}
  Probably also left as an exercise (not proved during the lecture).
\end{proof}
\begin{proof}[Proof. (Substitution lemma)] 
  Left as an exercise.
\end{proof}
\begin{proof}[Proof. (\circled{a} Generalized)] Proof by induction on $\Gamma \vdash e : \tau$.
\case{$\Gamma \vdash \true : bool$} \\
We have: 
\begin{description}
  \item $\gamma \models \Gamma$
\end{description}
We need to show:
\begin{description}
  \item $\SN{bool}{\gamma(\true)}$
\end{description}
If we do the substitution we just need to show $\SN{bool}{\true}$ which is true by definition of $\SN{bool}{\true}$.
\case{$\Gamma \vdash \false : bool$} similar to the \true{} case.
\case{\TVar}\\
We have: 
\begin{description}
  \item $\gamma \models \Gamma$
\end{description}
We need to show:
\begin{description}
  \item $\SN{\tau}{\gamma(x)}$
\end{description}
This case follows from the definition of $\Gamma \models \gamma$. We know that $x$ is well-typed, so it is in the domain of $\Gamma$. From the definition of $\Gamma \models \gamma$ we then get $\SN{\Gamma(x)}{\gamma(x)}$. From well-typedness of $x$ we have $\Gamma(x) = \tau$ which then gives us what we needed to show.
\case{$\Gamma \vdash \eif{e}{e_1}{e_2} : \tau$} left as an exercise.
\case{\TApp}\\
We have: 
\begin{description}
  \item $\gamma \models \Gamma$
\end{description}
We need to show:
\begin{description}
  \item $\SN{\tau}{\gamma(e_1 \; e_2)} \equiv \SN{\tau}{\gamma(e_1) \; \gamma(e_2)}$
\end{description}
By the induction hypothesis we have
\begin{align}
  &\SN{\tarrow{\tau_2}{\tau}}{\gamma(e_1)} \\
  &\SN{\tau_2}{\gamma(e_2)}
\end{align}
If we use the 3rd property of (1), $\forall e'. \; \SN{\tau_2}{e'} \implies \SN{\tau}{\gamma(e_1) \; e'}$, instantiated with (2), then we get $\SN{\tau}{\gamma(e_1) \; \gamma(e_2)}$ which is the result we need.
\case{\TAbs} \\
We have: 
\begin{description}
  \item $\gamma \models \Gamma$
\end{description}
We need to show:
\begin{description}
  \item $\SN{\tarrow{\tau_1}{\tau_2}}{\gamma(\tlabs{x}{\tau_1}{e})} \equiv \SN{\tarrow{\tau_1}{\tau_2}}{\tlabs{x}{\tau_1}{\gamma(e)}}$
\end{description}
Our induction hypothesis in this case reads:
\[
  \Gamma,x:\tau_1 \vdash e : \tau_2 \pand \gamma' \models \Gamma, x : \tau_1 \quad \implies \quad \SN{\tau_2}{\gamma'(e)}
\]
It suffices to show the following three things:
\begin{enumerate}
\item $\mtenv \vdash \tlabs{x}{\tau_1}{\gamma(e)} : \tarrow{\tau_1}{\tau_2}$
\item $\tlabs{x}{\tau_1}{\gamma(e)} \Downarrow$
\item $\forall e'. \SN{\tau_1}{e'} \implies \SN{\tau_2}{(\tlabs{x}{\tau_1}{\gamma(e)}) \; e'}$
\end{enumerate}
If we use the substitution lemma and push the $\gamma$ in under the $\lambda$-abstraction, then we get 1\footnote{Substitution has not been formally defined here, but one can find a sound definition in Pierce's Types and Programming Languages.}. 2 is okay as the lambda-abstraction is a value. 

It only remains to show 3. To do this we want to somehow apply the induction hypothesis for which we need a $\gamma'$ such that $\gamma' \models \Gamma, x:\tau_1$. We already have $\gamma$ and $\gamma \models \Gamma$, so our $\gamma'$ should probably have have the form $\gamma' = \gamma[x \mapsto v_?]$ for some $v_?$ of type $\tau_1$. Let us move on and see if any good candidates for $v_?$ present themselves.

Let $e'$ be given and assume $\SN{\tau_1}{e'}$. We then need to show $\SN{\tau_2}{(\tlabs{x}{\tau_1}{\gamma(e)}) \; e'}$. From $\SN{\tau_1}{e'}$ it follows that $e' \Downarrow v'$ for some $v'$. $v'$ is a good candidate for $v_?$ so let $v_? = v'$. From the forward part of the preservation lemma we can further conclude $\SN{\tau_1}{v'}$. We use this to conclude $\gamma[x\mapsto v'] \models \Gamma, x:\tau_1$ which we use with the assumption $\Gamma,x:\tau_1 \vdash e : \tau_2$ to instantiate the induction hypothesis and get $\SN{\tau_2}{\gamma[x\mapsto v'](e)}$.

Now consider the following evaluation:
\begin{align*}
  (\tlabs{x}{\tau_1}{\gamma(e)}) \; e' & \evaltos (\tlabs{x}{\tau_1}{\gamma(e)}) \; v' \\
                                       & \evalto \gamma(e)[v'/x] \equiv 
                                                   \gamma[x \mapsto v'](e)
\end{align*}
We already concluded that $e' \evaltos v'$ which corresponds to the first series of steps. We can then do a $\beta$-reduction to take the next step and finally we get something that is equivalent to $\gamma[x \mapsto v'](e)$. That is we have the evaluation
\[
(\tlabs{x}{\tau_1}{\gamma(e)}) \; e' \evaltos \gamma[x \mapsto v'](e)
\]
From \SN{\tau_1}{e'} we have $\mtenv \vdash e' : \tau_1$ and we already argued that $\mtenv \vdash \tlabs{x}{\tau_1}{\gamma(e)} : \tarrow{\tau_1}{\tau_2}$ so from the application typing rule we get $\mtenv \vdash (\tlabs{x}{\tau_1}{\gamma(e)}) \; e' : \tau_2$. We can use this with the above evaluation and the forward part of the preservation lemma to argue that every intermediate expressions in the steps down to $\gamma[x \mapsto v'](e)$ are closed and well typed.

If we use \SN{\tau_2}{\gamma[x\mapsto v'](e)} with $(\tlabs{x}{\tau_1}{\gamma(e)}) \; e' \evaltos \gamma[x \mapsto v'](e)$ and the fact that every intermediate step in the evaluation is closed and well typed, then we can use the backward reduction part of the $\SNPred$ preservation lemma to get \SN{\tau_2}{(\tlabs{x}{\tau_1}{\gamma(e)}) \; e'} which is the result we wanted.

\end{proof}
\subsection*{Exercises}
\begin{enumerate}
\item Prove $\SNPred$ preserved by forward/backward reduction.
\item Prove the substitution lemma.
\item Go through the cases of ``\circled{a} Generalized'' shown here by yourself.
\item Prove the if-case of ``\circled{a} Generalized''.
\item Extend the language with pairs and do the proofs. 
  \begin{enumerate}
  \item See how the clauses, we generally wanted our logical predicate to have, play out when we extend the logical predicate. Do we need to add anything for the third clause or does it work out without putting anything there like we did with the $bool$ case.
  \end{enumerate}
\end{enumerate}
\clearpage

\section*{Lecture 2, Type Safety}
The goal of this lecture was to proof type safety for STLC using logical relations. First we need to consider what type safety is. The classical mantra for type safety is ``Well-typed programs do not \emph{go wrong}.'' It depends on our language and type system what \emph{go wrong} means, but in our case a program has \emph{gone wrong} if it is stuck\footnote{If we consider language-based security for information flow control the notion of \emph{go wrong} would be that there is an undesired flow of information}. 

\subsection*{Type Safety for STLC}
\begin{stlctypesafety}[Type Safety for STLC]
  If $\mtenv \vdash e : \tau$ and $e \evaltos e'$ then $\val(e')$ or $\exists e''. \; e' \evalto e''$.
\end{stlctypesafety}
Traditionally type safety is proven by two lemmas, progress and preservation.
\begin{progress}[Progress]
  If $\mtenv \vdash e : \tau$ then $\val(e)$ or $\exists e'. \; e \evalto e'$.
\end{progress}
Progress is normally proved by induction on the typing derivation.
\begin{preservation}[Preservation]
  If $\mtenv \vdash e : \tau$ and $e \evalto e'$ then $\mtenv \vdash e' : \tau$.
\end{preservation}
Preservation is normally proved by induction on the evaluation.
Preservation is also known as \emph{subject reduction}. Progress and preservation talk about one step, so to prove type safety we have to do induction on the evaluation. Here we do not want to prove type safety the traditional way, we want to prove it using a logical predicate. We use a logical predicate rather than a logical relation, because type safety is a unary property.

The notation will here be changed compared to the one from lecture 1. We define the logical predicate in two parts: a value interpretation and an expression interpretation. We define the value interpretation as:
\begin{align*}
  \vpred{bool} & = \{ \true, \false \}\\
  \vpred{\tarrow{\tau_1}{\tau_2}} & = \{\tlabs{x}{\tau_1}{e} \vbar \forall v \in \curly{V}\sem{\tau_1}.\; e [v/x] \in \curly{E}\sem{\tau_2}\}
\end{align*}
We define the expression interpretation as:
\[
  \epred{\tau} = \{e \vbar \forall e'. \; e \evaltos e' \pand \irred(e') \implies e' \in \curly{V}\sem{\tau} \}
\]
\footnote{Notice that neither \vpred{\tau} nor \epred{\tau} requires well-typedness. Normally this would be a part of the predicate, but as the goal is to prove type safety we do not want it as a part of the predicate. In fact, if we did include a well-typedness requirement, then we would end up having to prove preservation for some of the proofs to go through.}The predicate $\irred$ is defined as:
\[
  \irred(e) \eqdef \not\exists e'. \; e \evalto e'
\]
%TODO: Exlpain in words what irred is.
The sets are defined on the structure of the types. \vpred{\tarrow{\tau_1}{\tau_2}} contains \epred{\tau_2}, but \epred{\tau_2} uses $\tau_2$ directly in \vpred{\tau_2}, so the definition is structurally well-founded. 

To prove type safety we first define a new predicate, $\safe$:
\[
  \safe(e) \eqdef \forall e' . \; e \evaltos e' \implies \val(e') \vee \exists e
''. \; e' \evaltos e''
\]
%TODO: Explain in words what safe is
We are now ready to prove type safety. Just like we did for strong normalization, we prove type safety in two steps:
\[
  \circled{a} \quad \mtenv \vdash e : \tau \implies e \in \epred{\tau}
\]
\[
  \circled{b} \quad e \in \epred{\tau} \implies \safe(e)
\]
Rather than proving \circled{a} directly we prove a more general theorem and get \circled{a} as a corollary. But we are not yet in a position to state the theorem. First we need to define define the interpretation of type environments:
\begin{align*}
  \gpred{\mtenv} & = \{ \emptyset \} \\
  \gpred{\Gamma,x:\tau} & = \{\gamma[x \mapsto v] \vbar 
    \gamma \in \gpred{\Gamma} \pand 
    v \in \vpred{\tau}\}
\end{align*}
Further we need to define semantic type safety:
\[
  \Gamma \models e : \tau \eqdef \forall \gamma \in \gpred{\Gamma} . \; \gamma(e) \in \epred{\tau}
\]
We can now define our generalized version of \circled{a}. 
\begin{btypesafety}[Fundamental Property]
  If $\Gamma \vdash e : \tau$ then $\Gamma \models e : \tau$
\end{btypesafety}
A theorem like this would typically be the first you prove after definiting a logical relation. The theorem says that every syntactic type safety implies semantic type safety. 

We also alter the \circled{b} part of the proof, so we prove
\[
  \mtenv \models e : \tau \implies \safe(e)
\]
\begin{proof}[Proof. (Altered \circled{b})]
Suppose $e \evaltos e'$ for some $e'$, then we need to show $\val(e')$ or $\exists e''. \; e' \evalto e''$. We procced by casing on whether or not $\irred(e')$:
\case{$\neg \irred(e')$} this case follows directly from the definition of $\irred$. $\irred(e')$ is defined as $\not \exists e''. \; e' \evalto e''$ and as the assumption is $\neg \irred(e')$ we get $\exists e''. \; e' \evalto e''$.
\case{$irred(e')$} By assumption we have $\mtenv \models e : \tau$. As the typing context is empty we choose the empty substitution and get $e \in \epred{\tau}$. We now use the definition of $e \in \epred{\tau}$ with what we supposed, $e \evaltos e'$, and the case assumption, $\irred(e')$, to conclude $e' \in \vpred{\tau}$. As $e'$ is in the value interpretation of $\tau$ we can conclude $\val(e')$. 
\end{proof}
\begin{proof}[Proof. (Fundamental Property)] Proof by induction on the typing judgment.
  \case{\TAbs} \\
We need to show $\Gamma \models \tlabs{x}{\tau_1}{e} : \tarrow{\tau_1}{\tau_2}$. First suppose $\gamma \in \gpred{\Gamma}$. Then it suffices to show
\[
  \gamma(\tlabs{x}{\tau_1}{e}) \in \epred{\tarrow{\tau_1}{\tau_2}} \equiv
  (\tlabs{x}{\tau_1}{\gamma(e)}) \in \epred{\tarrow{\tau_1}{\tau_2}}
\]
Now suppose that $\tlabs{x}{\tau_1}{\gamma(e)} \evaltos e'$ and $\irred(e')$. We then need to show $e' \in \vpred{\tarrow{\tau_1}{\tau_2}}$. Since $\tlabs{x}{\tau_1}{\gamma(e)}$ is a value it is irreducible, and we can conclude it took no steps. In other words $e' = \tlabs{x}{\tau_1}{\gamma(e)}$. So we need to show $\tlabs{x}{\tau_1}{\gamma(e)} \in \vpred{\tarrow{\tau_1}{\tau_2}}$. Now suppose $v \in \vpred{\tau_1}$ then we need to show $\gamma(e)[v/x] \in \epred{\tau_2}$.

Now let us keep the above proof goal in mind and consider the induction hypothesis:
\[
  \Gamma, x: \tau_1 \models e : \tau_2
\]
Instantiate this with $\gamma[x \mapsto v]$. We have $\gamma[x \mapsto v] \in \gpred{\Gamma, x : \tau_1}$ because we started by supposing $\gamma \in \gpred{\Gamma}$ and we also had $v \in \vpred{\tau_2}$. The instantiation gives us $\gamma[x \mapsto v] (e) \in  \epred{\tau_2} \equiv \gamma(e)[v/x] \in \epred{\tau_2}$. Now recall our proof goal, we now have what we wanted to show.
\case{\TApp} show this case as an exercise.

The remaining cases were not proved during the lecture.
\end{proof}
Now consider what happens if we add pairs to the language. 
%TODO: Clarify what additions to STLC are needed for pairs.
\begin{comment}
\begin{align*}
  &\fst <v_1,v_2> \evalto v_1 \\
  &\snd <v_1,v_2> \evalto v_2
\end{align*}
\end{comment}
We need to add a clause to the value interpretation:
\[
  \vpred{\tau_1 \times \tau_2} = \{<v_1,v_2> \vbar v_1 \in \vpred{\tau_1} \pand v_2 \in \vpred{\tau_2}\}
\]
There is nothing surprising in this addition to the value relation, and it should not be a challenge to show the pair case of the proofs.
%omitted: '3rd part of LR recipe is mostly about functions - Contravariance problem.'

If we extend our language with sum types. %that is
\[
e::= \dots \vbar \inl \; v \vbar \inr \; v \vbar \scase \; e \caseof \; \inl \; x => e_1 \quad \inr \; x => e_2
\]
Then we need to add the following clause to the value interpretation:
\[
  \vpred{\tau_1 + \tau_2} = \{\inl \; v \vbar v \in \vpred{\tau_1}\} \cup
                           \{\inr \; v \vbar v \in \vpred{\tau_2}\}
\]
It turns out this clause is sufficient. One might think that is is necessary to require the body to be in the expression relation, that is a requirements that looks something like $\forall e_1 \in \epred{\tau}$. This requirement will, however, give well-foundedness problems, as $\tau$ is not a structurally smaller type than $\tau_1 + \tau_2$. It may come as a surprise that we do not need to relate the expressions, as the slogan for logical relations is ``Related inputs to related outputs.''

\subsubsection*{Recursive Types}
In this subsection we will motivate and introduce recursive types. This will set the scene for lecture 3.

First consider the  following program in the untyped lambda calculus:
\[
  \Omega = (\labs{x}{x \; x}) \; (\labs{x}{x \; x})
\]
The interested reader can now try to evaluate the above expression. After a $\beta$-reduction and a substitution we end up with $\Omega$ again, so the evaluation of this expression diverges. Moreover, it is not possible to assign a type to $\Omega$ (again the interested reader may try to verify this by attempting to assign a type). It can hardly come as a surprise that it cannot be assigned a type, as we previously proved that the simply typed lambda calculus is strongly normalising, so if we could assign $\Omega$ a type it would not diverge.

We would like to have recursive types. For one they allow us to type things such as lists, trees, and streams. They do, however, also give us non-termination. In an ML like language a declaration of a tree type would look like this:
\begin{lstlisting}
  type tree = Leaf
            | Node of int * tree * tree
\end{lstlisting}
In Java we could define a tree class with an int field and fields for the subtrees:
\begin{lstlisting}
  class Tree {
    int value;
    Tree left, right;
  }
\end{lstlisting}
So we can define trees in our programming languages, but we cannot define them in the lambda calculus. So let us first consider what it is we want to define. we want a type that can either be a node or a leaf. A leaf can be represented by unit (as it here does not carry any information), and a node is the product of an int and two nodes. We put the two together with the sum type, as it can be either:
\[
  tree = 1 + (int * tree * tree)
\]
This is what we want, but we cannot specify this. We try to define $tree$, but $tree$ appears on the right hand side, it is self-referential. Now let us examine what we want. Instead of writing $tree$ we use a type variable $\alpha$:
\begin{align*}
  \alpha &= 1 + (int \times \alpha \times \alpha) \\
         &= 1 + (int \times (int \times \alpha \times \alpha) \times (int \times \alpha \times \alpha)) \\
  &\vdots
\end{align*}
All the sides of the above equations are equal, they are all trees. We could keep going and get an infinite system of equations. If we keep substituting the definition of $\alpha$ for $\alpha$ we keep getting bigger and bigger types. All of the types are trees, and all of them are finite. If we take the limit of this process, then we end up with an infinite tree, and that tree is the tree we conceptually have in our minds. So what we need is the fixedpoint of the above equation.

Let us define a recursive function who's fixedpoint we want to find:
\[
F = \lambda \alpha :: type. 1 + (int \times \alpha \times \alpha)
\]
We want the fixedpoint which by definition is $t$ such that
\[
  t = F(t)
\]
So we want
\[
  tree = F(tree)
\]
The fixed point of this function is written:
\[
  \mu \alpha.\; F(\alpha)
\]
Here $\mu$ is a fixedpoint type constructor. As the above is the fixedpoint, then by definition it should be equal to $F$ applied to it:
\[
  \mu \alpha.\; F(\alpha) = F(\mu \alpha.\; F(\alpha))
\]
Now let us make this look a bit more like types by substituting $F(\alpha)$ for $\tau$. 
\[
  \mu \alpha.\; \tau = F(\mu \alpha.\; \tau) 
\]
The right hand side is really just $\tau$ with $\mu \alpha. \; \tau$ substituted with $\tau$:
\[
  \mu \alpha.\; \tau = \tau[\mu \alpha. \; \tau / \alpha]
\]
We are going to introduce the recursive type $\mu \alpha.\; \tau$ to our language. When we have a recursive type we can shift our view to an expanded version $\tau[\mu \alpha. \; \tau / \alpha]$ and contract back to the original type. Expanding the type is called $\unfold$ and contracting is is called $\fold$.
\[
\begin{tikzpicture}[->,>=stealth',shorten >=1pt,auto,node distance=3cm,
  thick,main node/.style={rectangle}]

  \node[main node] (1) {$\mu\alpha.\; \tau$};
  \node[main node] (2) [right of=1] {$\tau[\mu \alpha. \; \tau / \alpha]$};

  \path[every node/.style={font=\sffamily\small}]
    (1) edge [bend left] node [above] {$\unfold$} (2)
    (2) edge [bend left] node [below] {$\fold$} (1);
\end{tikzpicture}
\]
With recursive types in hand we can now define our tree type:
\[
  tree \eqdef \mu \alpha. \; 1 + (int \times \alpha \times \alpha)
\]
When we want to work with this, we would like to be able to get under the $\mu$. Say we have $e : tree$ that is an expression $e$ with type $tree$, then we want to be able to say whether it is a leaf or a node. To do so we unfold the type to get the type where $\alpha$ has been substituted with the definition of $tree$ and the outer $\mu\alpha.$ has been removed. With the outer $\mu\alpha.$ gone we can match on the sum type to find out whether it is a leaf or a node. We can fold the type back to the original tree type, when we are done working with it.
\[
\begin{tikzpicture}[->,>=stealth',shorten >=1pt,auto,node distance=3cm,
  thick,main node/.style={rectangle}]

  \node[main node] (1) {$tree=\mu\alpha.\; 1+(int \times \alpha \times \alpha)$};
  \node[main node] (2) [right of=1,xshift=5cm] {$1+(int \times (\mu\alpha.\; 1+(int \times \alpha \times \alpha)) \times (\mu\alpha.\; 1+(int \times \alpha \times \alpha)))$};

  \path[every node/.style={font=\sffamily\small}]
    (1) edge [bend left=10] node [above] {$\unfold$} (2)
    (2) edge [bend left=10] node [below] {$\fold$} (1);
\end{tikzpicture}
\]
This kind of recursive types is called iso-recursive types, because there is an isomorphism between a $\mu\alpha. \; \tau$ and its unfolding $\tau[\mu\alpha.\; \tau / \alpha]$. 

STLC extended with recursive types is defined as follows:
\begin{align*}
  \tau &::= \dots \vbar \mu \alpha. \; \tau \\
  e    &::= \dots \vbar \fold \; e \vbar \unfold \; e \\
  v    &::= \dots \vbar \fold \; v\\
  E    &::= \dots \vbar \fold \; E \vbar \; \unfold \; E
\end{align*}
\[
\unfold \; (\fold \; v) \evalto v
\]
\[
\TFold \hspace{2cm} \TUnfold
\]
With this we could define the type of an integer list as:
\[
int\; list \eqdef \mu\alpha.\; 1 + (int \times \alpha)
\]
%TODO: Typing omega
\begin{comment}
\[
  \Omega = (\tlabs{x}{\mu\alpha.\; \tarrow{\alpha}{\tau}}{(\unfold \; x) \; x}) 
\]
\end{comment}
\subsection*{Exercises}
\begin{enumerate}
\item Prove the TApp case of the Fundamental Property
\end{enumerate}
\clearpage

\section*{Lecture 3, Step-Indexing}
This lecture was on extending the unary logical relation for type safety of simply typed lambda calculus to recursive types, introducing universal types, and finally a short introduction to contextual equivalence.

\subsection*{Simply typed lambda calculus extended with $\mu$}
In a naive first attempt to make the value interpretation we could write something like
\[
  \vpred{\mu\alpha. \; \tau} = \{\fold \; v \vbar \unfold \; (\fold \; v) \in \epred{\sub{\tau}{\mu\alpha.\;\tau}{\alpha}} \}
\]
We can simplify this slightly; first we use the fact that $\unfold \; (\fold \; v)$ reduces to $v$. Next, we use the fact that $v$ must be a value and the fact that we want $v$ to be in the expression interpretation of $\tau[\mu \alpha. \; \tau / \alpha]$. By unfolding the definition of the expression interpretation we conclude that it suffices to require $v$ to be in the value interpretation of the same type. We then end up with the following definition:
\[
  \vpred{\mu\alpha. \; \tau} = \{\fold \; v \vbar v \in \vpred{\sub{\tau}{\mu\alpha.\;\tau}{\alpha}} \}
\]
This gives us a well-foundedness issue. The value interpretation is defined by induction on the type, but $\sub{\tau}{\mu\alpha.\;\tau}{\alpha}$ is not a structurally smaller type than $\mu\alpha. \; \tau$. 

To solve this issue we index the interpretation by a natural number, $k$, which we write as follows:
\[
  \vpres{\tau} = \{v \vbar \dots \}
\]
If $v$ belongs to the interpretation, then this is read as ``$v$ belongs to the interpretation of $\tau$ for $k$ steps.'' We interpret this in the following way; given a value that we run run for $k$ or fewer steps as in the value is used in some program context for fewer than $k$, then we will never notice that it does not have type $\tau$. If we use the same value in a program context that wants to run for more than $k$ steps, then we might notice that it does not have type $\tau$ which means that we might get stuck. This gives us an approximate guarantee.

We use this as an inductive metric to make our definition well-founded, so we define the interpretation on induction on the step-index followed by an inner induction on the type structure. Let us start by adding the step-index to our existing value interpretation:
\begin{align*}
  \vpres{bool} &= \{\true,\false\} \\
  \vpres{\tarrow{\tau_1}{\tau_2}} &= \{\tlabs{x}{\tau_1}{e} \vbar \forall j \leq k. \; \forall v \in \vpres[j]{\tau_1}. \; \sub{e}{v}{x} \in \epres[j]{\tau_2} \}
\end{align*}
$\true$ and $\false$ are in the value interpretation of $bool$ for any $k$, so $\true$ and $\false$ will for any $k$ look like it has type $bool$. To illustrate how to understand the value interpretation of $\tarrow{\tau_1}{\tau_2}$ please consider the following time line:  \\
\begin{center}
\begin{tikzpicture}
    % draw horizontal line   
    \draw[->] (0,0) -- (8,0);

    % draw vertical lines
    \foreach \x in {0,3,5,8}
      \draw (\x cm,3pt) -- (\x cm,-3pt);

    % draw nodes
    \draw (-2,0) node[below=3pt] {  } node[above=6pt] {$\lambda$-time line};
    \draw (0,0) node[below=3pt] {$k$} node[above=3pt] {$(\tlabs{x}{\tau_1}{e}) \; e_2$};
    \draw (3,0) node[below=3pt] {$ j+1 $} node[above=3pt] {$(\tlabs{x}{\tau_1}{e})\; v $};
    \draw (4.3,0) node[below=3pt] {$   $} node[above=3pt] {$ \evalto $};    
    \draw (5,0) node[below=3pt] {$ j $} node[above=3pt] {$ \sub{e}{v}{x} $};
    \draw (8,0) node[below=3pt] {$ 0 $} node[above=3pt] {$  $};
    \draw (9,0) node[below=3pt] {$   $} node[above=3pt] { 'future' };
  \end{tikzpicture}
\end{center}
Here we start at index $k$ and as we run the program we use up steps until we at some point reach 0 and run out of steps. At step $k$ we are looking at a lambda. A lambda is used by applying it, but it is not certain that the application will happen right away. We only do a $\beta$-reduction when we try to apply a lambda to a value, but we might be looking at a context where we want to apply the lambda to an expressions, i.e.\ $(\tlabs{x}{\tau_1}{e})\; e_2$. We might use a bunch of steps to reduce $e_2$ down to a value, but we cannot say how many. So say that sometime in the future have fully evaluated $e_2$ to $v$ and say that we have $j+1$ steps left at this time, then we can do the $\beta$ reduction which gives us $\sub{e}{v}{x}$ at step $j$. % If we ever hit 0 steps, then all bets are off. the value can have any type.

We can now define the value interpretation of $\mu\alpha. \; \tau$:
\[
  \vpres{\mu\alpha.\; \tau} = \{\fold \; v \vbar \forall j < k. \; v \in \vpres[j]{\sub{\tau}{\mu\alpha.\;\tau}{\alpha}} \}
\]
This definition is like the one we previously proposed, but with a step-index. This definition is well-founded because $j$ is required to be \emph{strictly} less than $k$ and as we define the interpretation on induction over the step-index this is indeed well founded. We do not define a value interpretation for type variables $\alpha$, as we have no polymorphism yet. The only place we have a type variable at the moment is in $\mu\alpha. \; \tau$, but in the interpretation we immediately close of the $\tau$ under the $\mu$, so we will never encounter a free type variable.

Finally, we define the expression interpretation:
\[
  \epres{\tau} = \{e \vbar \forall j < k. \; \forall e'. \; e \evaltos[j] e' \pand \irred(e') \; \implies \; e' \in \vpres[k-j]{\tau}\}
\]
To illustrate what is going on here consider the following time line: \\
\begin{center}
\begin{tikzpicture}
    % draw horizontal line   
    \draw[->] (0,0) -- (4,0);

    % draw vertical lines
    \foreach \x in {0,2,4}
      \draw (\x cm,3pt) -- (\x cm,-3pt);

    % draw nodes
    \draw (0,0) node[below=3pt] {$k$} node[above=3pt] {$e$};
    \draw (1,0) node[below=3pt] {$ $} node[above=3pt] {$\evalto \evalto \evalto \evalto$};
    \draw (2,0) node[below=3pt] {$k-j$} node[above=3pt] {$e'$};
    \draw (4,0) node[below=3pt] {$0$} node[above=3pt] {$  $};

    % brace
    \draw [decorate,decoration={brace,amplitude=10pt,mirror}]
    (0,-0.6) -- (2,-0.6) node [black,midway,yshift=-0.5cm] 
          {\footnotesize $j$};
\end{tikzpicture}
\end{center}
We start with an expression $e$, then we take $j$ steps and get to expression $e'$. At this point if $e'$ is irreducible, then we want it to belong to the value interpretation of $\tau$ for $k-j$ steps. We use a strict inequality because we do not want to hit 0 steps. If we hit 0 steps, then all bets are off.%TODO: why are all bets off?

We also need to lift the interpretation of type environments to step-indexing:
\begin{align*}
  \gpres{\mtenv} & = \{\emptyset \} \\
  \gpres{\Gamma, x : \tau} & = \{ \gamma[x \mapsto v] \vbar \gamma \in \gpres{\Gamma} \pand v \in \vpres{\tau} \}
\end{align*}
Finally we are in a position to lift the definition of semantic type safety to one with step-indexing.
\[
  \Gamma \models e : \tau \eqdef \forall k \geq 0. \; \forall \gamma \in \gpres{\Gamma} \; \implies \gamma(e) \in \epres{\tau}
\]
To actually prove type safety we do it in two steps. First we state and prove the fundamental theorem:
\begin{stlcmufundprop}[Fundamental property] ~\\
  If $\Gamma \vdash e : \tau$ then $\Gamma \models e : \tau$.
\end{stlcmufundprop}
When we have proven the fundamental theorem we prove that it entails type safety.
\[
\circled{b} \quad \mtenv \models e : \tau \implies \safe(e)
\]
Thanks to the way we defined the logical predicate this second step should be trivial to prove.

To actually prove the fundamental theorem, which is the challenging part, we need to prove a monotonicity lemma:
\begin{monotonicity}[Monotonicity] ~\\
  If $v\in \vpres{\tau}$ and $j \leq k$  then $v \in \vpres[j]{\tau}$.
\end{monotonicity}
\begin{proof}
The proof is by case on $\tau$.
\case{$\tau = bool$}
assume $v \in \vpres{bool}$ and $j \leq k$, we then need to show $v \in \vpres[j]{bool}$. As $v \in \vpres{bool}$ we know that either $v= \true$ or $v=\false$. If we assume $v=\true$, then we immediately get what we want to show, as $\true$ is in $\vpres[j]{bool}$ for any $j$. Likewise for the case $v=\false$.
\case{$\tau = \tarrow{\tau_1}{\tau_2}$}
assume $v \in \vpres{\tarrow{\tau_1}{\tau_2}}$ and $j \leq k$, we then need to show $v \in \vpres[j]{\tarrow{\tau_1}{\tau_2}}$. As $v$ is a member of $\vpres{\tarrow{\tau_1}{\tau_2}}$ we can conclude that $v = \tlabs{x}{\tau_1}{e}$ for some $e$. By definition of $v \in \vpres[j]{\tarrow{\tau_1}{\tau_2}}$ we need to show $\forall i \leq j. \forall v' \in \vpres[i]{\tau_1}.\; \sub{e}{v'}{x} \in \epres[i]{\tau_2}$. Suppose $i \leq j$ and $v' \in \vpres[i]{\tau_1}$, we then need to show $\sub{e}{v'}{x} \in \epres[i]{\tau_2}$.

By assumption we have $v \in \vpres{\tarrow{\tau_1}{\tau_2}}$ which gives us $\forall n \leq k. \; \forall v' \in \vpres[n]{\tau_1}.\; \sub{e}{v'}{x} \in \epres[n]{\tau_2}$. From $j \leq k$ and $i \leq j$ we get $i \leq k$ by transitivity. We use this with $v' \in \vpres[i]{\tau_1}$ to get $\sub{e}{v'}{x} \in \epres[i]{\tau_2}$ which is what we needed to show.
\case{$\tau = \mu \alpha.\; x$}
assume $v \in \vpres{\mu\alpha. \; \tau}$ and $j \leq k$, we then need to show $v \in \vpres[j]{\mu\alpha. \; \tau}$. From $v$'s assumed membership of the value interpretation of $\tau$ for $k$ steps we conclude that there must exist a $v'$ such that $v = \fold \; v'$. If we suppose $i<j$, then we need to show $v' \in \vpres[i]{\subst{\tau}{\mu\alpha.\; \tau}{\alpha}}$. From $i<j$ and $j \leq k$ we can conclude $i < k$ which we use with $\forall n < k.\; v' \in \vpres[n]{\subst{\tau}{\mu\alpha.\; \tau}{\alpha}}$, which we get from $v \in \vpres{\mu\alpha. \; \tau}$, to get $v' \in \vpres[i]{\subst{\tau}{\mu\alpha.\; \tau}{\alpha}}$.
\end{proof}
\begin{comment}
  \begin{lemma}[Substitution]
    Let $e$ be syntactically well-formed term, let $v$ be a closed value and let $\gamma$ be a substitution that map term variables to closed values, and let $x$ be a variable not in the domain of $\gamma$, then
    \[
    \extsub{\gamma}{x}{v}(e) = \subst{\gamma(e)}{x}{v}
    \]
  \end{lemma}
\end{comment}
\begin{proof}[Proof (Fundemental Property)]
Proof by induction over the typing derivation.
\case{\TFold} \\~
We need to show 
\newcommand{\mat}{\ensuremath{\mu\alpha.\tau}}
\[
  \Gamma \models \fold \; e : \mat
\]
So suppose we have $k \geq 0$ and $\gamma \in \gpres{\mat}$, then we need to show $\gamma(\fold\; e) \in \epres{\mat}$ which amounts to showing $\fold\; \gamma(e) \in \epres{\mat}$.

So suppose that $j<k$ and that $\fold\; \gamma(e) \evaltos[j] e'$ and $\irred(e')$, then we need to show $e' \in \vpres[k-j]{\mat}$. As we have assumed that $\fold\; \gamma(e)$ reduces down to something irreducible and the operational semantics of this language are deterministic we know that $\gamma(e)$ must have evaluated down to something irreducible. We therefore know that $\gamma(e) \evaltos[j_1] e_1$ where $j_1 \leq j$ and $\irred(e_1)$.
Now we use our induction hypothesis:
\newcommand{\tsub}{\ensuremath{\sub{\tau}{\mat}{\alpha}}}
\[
  \Gamma \models e : \tsub
\]
We instantiate this with $k$ and $\gamma \in \gpres{\Gamma}$ to get $\gamma(e) \in \epres{\tsub}$. Which we then can instantiate with $j_1$ and $e_1$ for get $e_1 \in \vpres[k-j_1]{\tsub}$. Now let us take a step back and see what happened: We started with a $\fold\; \gamma(e)$ which took $j_1$ steps to $\fold\; e_1$. We have just shown that this $e_1$ is actually a value as it is in the value interpretation of $\vpres[k-j_1]{\tsub}$. To remind us $e_1$ is a value let us henceforth refer to it as $v_1$. We further know that $\fold\; \gamma(e)$ reduces to $e'$ in $j$ steps and that $e'$ was irreducible. $\fold\; v_1$ is also irreducible as it is a value and as our language is deterministic it must be the case that $e' = \fold\; v_1$ and thus $j = j_1$. Our proof obligation was to show $e' = \fold \; v_1 \in \vpres[k-j]{\mat}$ to show this suppose we have $l < k-j$ (this also gives us $l < k-j_1$ as $j = j_1$). We then need to show $v_1 \in \vpres[l]{\tsub}$, we obtain this result from the monotonicity lemma using $\vpres[k-j_1]{\tsub}$ and $l < k-j_1$.
\end{proof}

The list example from the previous lecture used the sum type. Sums are a straight forward extension of the language. The extension of the value interpretation would be:
\[
  \vpres{\tau_1 + \tau_2} = \{\inl \; v_1 \vbar v_1 \in \vpres{\tau_1}\} \cup 
                            \{\inr \; v_2 \vbar v_2 \in \vpres{\tau_2}\}
\]
We can use $k$ directly or $k$ decremented by one. It depends on whether we want casing to take up a step. Either way the definition is well-founded. 

\subsection*{Universal Types}
Now we shift focus from type safety and termination to program equivalence. To motivate the need for arguing about program equivalence we first introduce universal types.

Say we have a function that sorts integer lists:
\[
  sortint \; : \; \tarrow{list \; int}{list \; int}
\]
$sortint$ takes a list of integers and returns a sorted version of that list. Say we now want a function that sorts lists of strings, then instead of implementing a separate function we could factor out the code responsible for sorting and have just one function. The type signature of such a generic sort function is:
\[
  sort \; : \; \forall \alpha.\; \tarrow{(list \; \alpha) \times (\tarrow{\alpha \times \alpha}{bool} )}{list \; \alpha}
\]
$sort$ takes a type, a list of elements of this type, and a comparison function that compares to elements of the type argument and returns a list sorted according to the comparison function. An example of an application of this function could be
\[
  sort \; [int] \; (3,7,5) \; <
\]
Whereas sort instantiated with the $string$ type, but given an integer list would not be a well typed instantiation.
\[
  sort \; [string] \; \cancelto{("a","c","b")}{(3,7,5)} \; string<
\]
Here the application with the list $(3,7,5)$ is not well typed, but if we instead use a list of strings, then it type checks. 

We want to extend the simply typed lambda calculus with functions that abstract over types in the same way lambda abstractions, $\tlabs{x}{\tau}{e}$, abstract over terms. We do that by introducing a type abstraction:
\[
  \tLabs{e}
\]
This function abstracts over the type $\alpha$ which allows $e$ to depend on $\alpha$.

\subsubsection*{System F (Simply Typed Lambda Calculus With Universal Types)}
\begin{align*}
  \tau &::= \dots \vbar \forall \alpha. \; \tau \\
  e    &::= \dots \vbar \tLabs{e} \vbar e[\tau] \\
  v    &::= \dots \vbar \tLabs{e}\\
  E    &::= \dots \vbar E[\tau] 
\end{align*}
\[
  (\tLabs{e})[\tau] \evalto e[\tau/\alpha]
\]
Type environment\footnote{We do not annotate $\alpha$ with a kind, as we only have one kind in this language.} (The type environment is assumed to consist of distinct type variables. For instance, the environment $\Delta,\alpha$ is only well formed if $\alpha \not\in \dom(\Delta)$).
\[
  \Delta ::= \mtenv \vbar \Delta,\; \alpha
\]
With the addition of type environments of type variables we our typing judgments now have the following form:
\[
  \Delta,\Gamma \vdash e : \tau
\]
We now need a notion of well-formed types. If $\tau$ is well formed with respect to $\Delta$, then we write:
\[
  \Delta \vdash \tau
\]
We do not include the formal rules here, but they amount to $\FTV(\tau) \subseteq \Delta$, where $\FTV(\tau)$ is the set of free type variables in $\tau$.

We further introduce a notion of well formed environments. An environment is well formed if all the types that appear in the range of $\Gamma$ are well formed.
\[
  \Delta \vdash \Gamma \eqdef \forall x \in \dom(\Gamma). \; \Delta \vdash \Gamma(x)
\]
For any type judgment $\Delta,\Gamma \vdash e : \tau$ we have as an invariant that $\tau$ is well-formed in $\Delta$ and $\Gamma$ is well formed in $\Delta$. The old typing system modified to use the new form of the typing judgment looks like this:
\[
  \FTFalse
\hspace{1cm}
  \FTTrue
\]
\[
  \FTVar
\hspace{1cm}
  \FTIf
\]  
\[
  \FTAbs 
\hspace{1cm}
  \FTApp
\]
Notice that the only thing that has changed is that $\Delta$ has been added to the environment in the judgments. We further extend the typing rules with the following two rules to account for our new language constructs:
\[
  \FTtAbs
\hspace{1cm}
  \FTtApp
\]
\subsubsection*{Properties of System-F}
In System-F certain types reveal the behavior of the functions with that type. Let us consider terms with the type $\forall \alpha.\; \tarrow{\alpha}{\alpha}$. Recall from the motivation in lecture 1 that this had to be the identity function. We can now phrase this as a theorem:
\begin{theorem}
  If $\mtenv ; \mtenv \vdash e : \forall \alpha. \; \tarrow{\alpha}{\alpha}$,\\ $\mtenv \vdash \tau$, and\\ $\mtenv; \mtenv \vdash v : \tau$,\\ then $e[\tau]\; v \evaltos v$
\end{theorem}
This is a free theorem in this language. Another free theorem that was mentioned in the motivation of lecture 1 was about expressions with type $\forall \alpha. \tarrow{\alpha}{bool}$. Here all expressions with this type had to be constant functions. We can also phrase this as a theorem
\begin{theorem}
  If $\mtenv \vdash \tau$, $\mtenv \vdash v_1 : \tau$,\\ and $\mtenv \vdash v_1 : \tau$,\\ then $\ctxeq{e[\tau] \; v_1}{e[\tau]\; v_2}$.
\end{theorem}
Or in a slightly more general fashion where we allow different types:
\begin{theorem}
  If $\mtenv \vdash \tau$, \\ $\mtenv \vdash \tau'$, \\$\mtenv \vdash v_1 : \tau$,\\ and $\mtenv \vdash v_1 : \tau'$,\\ then $\ctxeq{e[\tau]\; v_1}{e[\tau']\; v_2}$.
\end{theorem}
\footnote{We have not yet defined $\ctxeq{}{}$ so for now just treat it as the two programs are equivalent without thinking too much about what equivalence means.}We get these free theorems because the functions have no way of inspecting the argument as they do not know what type it is. As the function has to treat its argument as an unknown ``blob'' it has no choice but to return the same value every time.

The question now is: ``how do we prove these free theorems?'' The two last theorems both talk about program equivalence which we prove using logical relations. The first theorem did not mention equivalence, but the proof technique of choice is still a logical relation.

\subsection*{Contextual Equivalence}
To define a contextual equivalence we first define the notion of a program context. A program context is a complete program with exactly one hole in it. It is defined as follows:
\begin{align*}
  C ::= &\;  [\cdot] \\
        &\vbar \eif{C}{e}{e}  \\
        &\vbar \eif{e}{C}{e} \\
        &\vbar \eif{e}{e}{C} \\
        &\vbar \tlabs{x}{\tau}{C} \\
        &\vbar C \; e \\
        &\vbar e \; C \\
        &\vbar \tLabs{C} \\
        &\vbar C[\tau]
\end{align*}
We need a notion of context typing. For simplicity we just introduce it for simply typed lambda calculus. The context typing is written as:
\[
C:\; (\Gamma \vdash \tau) \implies (\Gamma' \vdash \tau') \implies \Gamma \vdash e : \tau \implies \Gamma' \vdash C[e] : \tau'
\]
This means that for any expression $e$ of type $\tau$ under $\Gamma$ if we embed it into $C$, then the type of the embedding is $\tau'$ under $\Gamma'$.

Informally we want contextual equivalence to say that no matter what program context we embed either of the two expressions in, it gives the same result. This is also called as observational equivalence as the program context is unable to observe any difference no matter what expression we embed in it. We can of course not plug an arbitrary term into the hole, so we annotate the equivalence with the type of the hole which means that the two contextual equivalent expressions have to have that type.
\[
  \Delta; \Gamma \vdash \ctxeq{e_1}{e_2} : \tau \eqdef \forall C.\; (\Delta ; \Gamma \vdash \tau ) \implies (\mtenv ; \mtenv \vdash \tau') \implies (C[e_1] \Downarrow v \iff C[e_2] \Downarrow v)
\]
This definition assumes that $e_1$ and $e_2$ has type $\tau$ under the specified contexts.

Contextual equivalence is handy because we want to be able to reason about the equivalence of two implementations. Say we have two implementations of a stack, one is implemented using an array the other using a list. If we can show that the two implementations are contextual equivalent, then we can use the more efficient one over the less efficient one and know that the complete program will behave the same.

In the next lecture we will introduce a logical relation such that
\[
  \Delta ; \Gamma \vdash \lreq{e_1}{e_2} : \tau \implies \text{contextual equivalence \ctxeq{}{}} 
\]
That is we want to show that the logical relation is sound with respect to contextual equivalence. 

If we can prove the above soundness, then we can state our free theorems with $\lreq{}{}$ rather than $\ctxeq{}{}$ and get the same result if we can prove the logical equivalence. We would like to do this as it is difficult to directly prove two things are contextual equivalent. A direct proof has to talk about all possible program contexts which we could do using induction, but the lambda-abstraction case turns out to be difficult. This motivates the use of other proof methods where using a logical relation is one of them.

\subsection*{Exercises}
\begin{enumerate}
\item Do the lambda and application case of the \emph{Fundamental Property} theorem.%Specify in what proof, probably fundemental property
\item Try to prove the \emph{monotonicity} lemma where the definition of the value interpretation has been adjusted with:
\[
\vpres{\tarrow{\tau_1}{\tau_2}} = \{\tlabs{x}{\tau_1}{e} \vbar \; \forall v \in \vpres{\tau_1}. \; \sub{e}{v}{x} \in \epres{\tau_2} \}
\]
This will fail, but it is instructive to see how it fails.
\end{enumerate}


%induction on step index followed by inner induction on the type structure.
%when an expression looks like it has type tau.
\clearpage

%\section*{Lecture 4}
\subsection*{A Logical Relation for System F}
Now we need to build a logical relation for System F. With this logical relation we would like to be able to prove the free theorems from lecture 3. Our value interpretation will now consist of pairs, as we are defining a relation, so it will have the following form:
\[
  \vpred{\tau} = \{(v_1,v_2) \vbar \dots \}
\]
In our value interpretation we require $v_1$ and $v_2$ to be closed and well typed, but for succinctness we do not write this in the definitions below. 
Let us try to naively build the logical relation the same way we build the logical predicates:
\begin{align*}
  \vpred{bool}                 & = \{ (\true,\true), (\false,\false) \} \\
  \vpred{\tarrow{\tau}{\tau'}} & = \{ (\tlabs{x}{\tau}{e_1},\tlabs{x}{\tau}{e_2}) \vbar \forall (v_1,v_2) \in \vpred{\tau}. \; (\subst{e_1}{v_1}{x},\subst{e_2}{v_2}{x}) \in \epred{\tau'} \}
\end{align*}
The value interpretation of the function type is defined based on the slogan for logical relations ``related input to related output.'' If we had chosen to use equal input rather than related, then our definition would be more restrictive than necessary.

We did not define a value interpretation for type variables in lecture 3, so let us push on with out defining that one.

The next type is $\forall \alpha. \; \tau$. When we define the value interpretation we consider the elimination forms which in this case is type application. Before we proceed let us consider one of the free theorems from lecture 3 that we wanted to be able to prove:
\begin{theorem}
  If $\mtenv \vdash \tau$, 
  $\mtenv \vdash \tau'$, 
  $\mtenv \vdash v_1 : \tau$, 
  and $\mtenv \vdash v_1 : \tau'$, 
  then $\ctxeq{e[\tau]\; v_1}{e[\tau']\; v_2} : bool$.
\end{theorem}
There are some important points to notice in this free theorem. First of all, we want to be able to apply $\Lambda$-terms to different types, so in our value interpretation we will have to pick to different types. Further, normally we pick related expressions, so it would probably be a good idea to pick related types. We do, however, not have a notion of related types, and in the theorem there is no relation between the two types used, so we probably should not relate them. With these points in mind we can make a first attempt at defining the value interpretation of $\forall \alpha. \; \tau$:
\[
  \vpred{\forall \alpha. \; \tau} = \{(\tLabs{e_1}, \tLabs{e_2}) \vbar \forall \tau_1,\tau_2. \; (\subst{e_1}{\tau_1}{\alpha},\subst{e_2}{\tau_2}{\alpha}) \in \epred{\subst{\tau}{?}{\alpha}} \}
  \]
Now the question is what to type to relate the two expressions under. We need to substitute $?$ for something, but if we use either $\tau_1$ or $\tau_2$, then the well-typedness requirement will be broken. What we choose to do is to leave $\tau$ as it is and not do the substitution. When we do this, then we need to remember what types we picked in the left and right part of the pair. To do this we use a relational substitution:
\[
  \rho = \{ \alpha_1 \mapsto (\tau_{11},\tau_{12} ), \dots \}
\]
Which we parameterise the interpretations with.
\[
  \vprep{\forall \alpha.\; \tau} = \{(\tLabs{e_1},\tLabs{e_2}) \vbar \forall \tau_1 , \tau_2. \; (\subst{e_1}{\tau_1}{\alpha},\subst{e_2}{\tau_2}{\alpha}) \in \eprep[\extsub{\rho}{\alpha}{(\tau_1,\tau_2)}]{\tau} \}
\]
We need to parameterise the entire logical relation with the relational substitution, otherwise we will not know what type to pick when we interpret the polymorphic type. Which leads us to the next issue. We are now interpreting types with free type variables, so we need to have a value interpretation of type $\alpha$. It will look something like
\[
\vprep{\alpha} = \{(v_1,v_2) \vbar \rho(\alpha) = (\tau_1,\tau_2) \dots\}
\]
We need to say that the values are related, but the question is how to relate them. To figure this out we again look to the free theorem. In the free theorem the two values are related at the argument type we choose. We therefore pick a relation on these types when we pick the types. We remember the relation we pick in the relational substitution, so we reach our final definition of the value interpretation of $\forall \alpha. \; \tau$:
\[
  \vprep{\forall \alpha.\; \tau}  = \{(\tLabs{e_1},\tLabs{e_2}) \vbar 
  \forall \tau_1 , \tau_2, R. \; 
    R \in \Rel[\tau_1,\tau_2].\;
      (\subst{e_1}{\tau_1}{\alpha},\subst{e_2}{\tau_2}{\alpha}) \in \eprep[\extsub{\rho}{\alpha}{(\tau_1,\tau_2,R)}]{\tau} \} 
\]
We do not require much of the relation. It has to be a set of pairs of values, and both values in every pair in the relation have to be closed and well typed under the corresponding type. So we define $\Rel[\tau_1,\tau_2]$ as:
\[
\Rel[\tau_1,\tau_2] = \{R \in \curly{P}(Val \times Val) \vbar \forall (v_1,v_2) \in R. \; \mtenv \vdash v_1 : \tau_1 \pand \mtenv \vdash v_2 : \tau_2 \}
\]
In the interpretation of $alpha$ we require the values to be related under the relation we choose in the value interpretation of $\forall \alpha. \; \tau$:
\[
  \vprep{\alpha} = \{ (v_1,v_2) \vbar \rho(\alpha) = (\tau_1,\tau_2,R) \pand (v_1,v_2) \in R \}
\]
For convenience we introduce the following notation for projection in $\rho$. Given
\[
  \rho = \{\alpha_1 \mapsto (\tau_{11},\tau_{12},R_1), \alpha_2 \mapsto (\tau_{21},\tau_{22},R_2),\; \dots\; \}
\]
Define the following projections:
\begin{align*}
  \rho_1 & = \{\alpha_1 \mapsto \tau_{11}, \; \alpha_2 \mapsto \tau_{21},\; \dots \;\} \\
  \rho_2 & = \{\alpha_1 \mapsto \tau_{12}, \; \alpha_2 \mapsto \tau_{22},\; \dots \;\} \\
  \rho_R & = \{\alpha_1 \mapsto R_1, \; \alpha_2 \mapsto R_2,\; \dots \;\} 
\end{align*}
Notice that $\rho_1$ and $\rho_2$ now are type substitutions, so we write $\rho_1(\tau)$ to mean $\tau$ where all the variables mentioned in the substitution has been substituted. We can now write the value interpretation for type variables in a more succinct way:
\[
  \vprep{\alpha} = \rho_R(\alpha)
\]
%TODO: include something like: ``When we use our logical relation to prove results such as the free theorem above, then we get to choose the relation $R$ which means that we get to choose what values are related at $\alpha$!''? and ``The power of parametricity is that you pick types for $\alpha$ as well as the notion values of those types are related under.
%TODO: further: ``If we want to prove something about a term with type $\forall \alpha. \; \tau$, then we get to pick the the relation R. Whereas if we want to prove a term is of a polymorphic type, then we have the relation as a proof obligation and have to show it for an arbitrary R.
We need to add $\rho$ to the other parts of the value interpretation as well. Moreover, as we now interpret open types we require the pairs of values in the relation to be well typed under the type closed of using the relational substitution. So all value interpretations have the form
\[
  \vprep{\tau} = \{(v_1,v_2) \vbar \mtenv ; \mtenv \vdash v_1 : \rho_1(\tau) \pand \mtenv ; \mtenv \vdash v_2 : \rho_2(\tau) \dots \}
\]
Where ``$\dots$'' is replaced with the remaining conditions. We further need to close of the type of the variable in functions, so our value interpretations end up as:
\begin{align*}
  \vprep{bool}                 & = \{ (\true,\true), (\false,\false) \} \\
  \vprep{\tarrow{\tau}{\tau'}} & = \{ (\tlabs{x}{\rho_1(\tau)}{e_1},\tlabs{x}{\rho_2(\tau)}{e_2}) \vbar \forall (v_1,v_2) \in \vprep{\tau}. \; (\subst{e_1}{v_1}{x},\subst{e_2}{v_2}{x}) \in \eprep{\tau'} \}
\end{align*}
We define our interpretation of expressions as follows:
\begin{align*}
  \eprep{\tau} = \{(e_1,e_2) \vbar & \mtenv ; \mtenv \vdash e_1 : \rho_1(\tau) \pand \\
                                   & \mtenv ; \mtenv \vdash e_2 : \rho_2(\tau) \pand \\
                                   &  \exists v_1,v_2.\; e_1 \evaltos v_1 \pand
                                            e_2 \evaltos v_2 \pand
                                            (v_1,v_2) \in \vprep{\tau} \}
\end{align*}
We now need to give an interpretation of the contexts $\Delta$ and $\Gamma$:
\begin{align*}
  \dpred{\mtenv}         &= \{\emptyset \} \\
  \dpred{\Delta, \alpha} &= \{\extsub{\rho}{\alpha}{(\tau_1,\tau_2,R)} \vbar 
                                 \gamma \in \dpred{\Delta} \pand 
                                 R \in \Rel[\tau_1,\tau_2] \} \\
  \gprep{\mtenv}         &= \{\emptyset \} \\
  \gprep{\Gamma, x:\tau} &= \{\extsub{\gamma}{x}{(v_1,v_2)} \vbar 
                                 \gamma \in \gprep{\Gamma} \pand
                                 (v_1,v_2) \in \vprep{\tau}\}
\end{align*}
We need the relational substitution in the interpretation of $\Gamma$, because $\tau$ might have free type variables in it now.
%In \gpred{\Gamma, x:\tau} \tau may have free type variable in it which is why we need \rho. recall \Delta ; \Gamma \vdash e : \tau (?)
We introduce a convenient notation for the projections of $\gamma$ similar to the one we did for $\rho$: %TODO: Write something about the fact that we now want to close of pairs of expressions, so we get a substitution on pairs.
\[
  \gamma = \{\alpha_1 \mapsto (v_{11},v_{12}), \alpha_2 \mapsto (v_{21},v_{22}),\; \dots\; \}
\]
Define the projections as follows:
\begin{align*}
  \gamma_1 & = \{\alpha_1 \mapsto v_{11}, \; \alpha_2 \mapsto v_{21},\; \dots \;\} \\
  \gamma_2 & = \{\alpha_1 \mapsto v_{12}, \; \alpha_2 \mapsto v_{22},\; \dots \;\}
\end{align*}
%write \approx for \lreq{}{}
We are now ready to define when two terms are related. We define it in a similar way to the logical predicate we already have defined. First we pick a $\rho$ and a $\gamma$ to close of the terms, and then we require them to be related under the expression interpretation of the type in question:
\renewcommand{\lreq}[2]{\equivalence{#1}{}{#2}}
\begin{align*}
  \Delta ; \Gamma \vdash \lreq{e_1}{e_2} : \tau \eqdef & \Delta ; \Gamma \vdash e_1 : \tau \pand\\
                                                      & \Delta ; \Gamma \vdash e_2 : \tau \pand \\
                                                      & \forall \rho \in \dpred{\Delta}. \; \forall \gamma \in \gprep{\Gamma}. \; (\rho_1(\gamma_1(e_1)),\rho_2(\gamma_2(e_2))) \in \eprep{\tau}
\end{align*}
Now that we have defined our logical relation the first thing we want to do is to prove the fundamental property:
\begin{fundamentalprop}[Fundamental Property]
  If $\Delta; \Gamma \vdash e : \tau$ then $\Delta; \Gamma \vdash \lreq{e}{e} : \tau$
\end{fundamentalprop}
This theorem may seem a bit mundane, but it is actually quite strong. If we look into the definition of the logical relation, then $\Delta$ and $\Gamma$ can be seen as maps of placeholders that needs to be replaced in the expression. So when we choose a $\rho$ and $\gamma$ we may pick different types and terms to input in the expression. Closing the expression of can then give us two very different programs. 

%TODO add (?): ``In some presentations this is also known as the parametricity lemma. It may even be stated with out the short-hand notation we use here.''
We could prove the theorem directly by induction over the typing derivation, but we will instead prove it by means of compatibility lemmas.
%''Parametricity'' \Delta; \Gamma <- Placeholders for input. When we close of the terms we get two different programs.
%Prove directly via induction on typing judgement. One case per typing rule.
\subsubsection*{Compatibility Lemmas}
We state a compatibility for each of the typing rules we have. Each of the lemmas will correspond to a case in the induction proof of the Fundamental Property, so the that theorem will follow directly from the compatibility lemmas. We state the compatibility like rules to highlight the connection to the typing rules. The premises of the lemma is over the horizontal line and the conclusion is below:
\begin{enumerate}
\item $\Gamma ; \Delta \vdash \lreq{\true}{\true} : bool$ % \approx \true ``added to typing rule'' (?)
\item $\Gamma ; \Delta \vdash \lreq{\false}{\false} : bool$
\item $\Gamma ; \Delta \vdash \lreq{x}{x} : \Gamma(x)$
\item $\inferrule*{\Delta;\Gamma \vdash \lreq{e_1}{e_2} : \tarrow{\tau'}{\tau} \and
                   \Delta;\Gamma \vdash \lreq{e_1'}{e_2'} : \tau'}
                  {\Delta;\Gamma \vdash \lreq{e_1 \; e_1'}{e_2 \; e_2'} : \tau}$
%slightly more general than we need, different things on both sides, gives us FP, but we can use them to prove more.
\item $\inferrule*{\Delta; \Gamma, x:\tau \vdash \lreq{e_1}{e_2} : \tau'}
                  {\Delta; \Gamma \vdash \lreq{\tlabs{x}{\tau}{e_1}}{\tlabs{x}{\tau}{e_2}} : \tarrow{\tau}{\tau'}}$
\item $\inferrule*{\Delta; \Gamma \vdash \lreq{e_1}{e_2} : \forall \alpha . \tau \and
                   \Delta \vdash \tau'}
                  {\Delta; \Gamma \vdash \lreq{e_1[\tau']}{e_2[\tau']} : \subst{\tau}{\tau'}{\alpha}}$
%TODO: Add if? (then delete the text that says it has been omitted
\end{enumerate}
The rule for if has been omitted here. Notice that some of the lemmas are more general then what we actually need. Take for instance the compatibility lemma for expression application. Here we really just needed to have the same thing on both sides of the equivalence, but we do not we have something slightly more general. It turns out that the slightly more general version helps when we want to prove that the logical relation is sound with respect to contextual equivalence.

We will only prove the compatibility lemma for type application. To do so we are going to need the following lemma:
\begin{lemma}[Compositionality]
  Let $\Delta \vdash \tau'$, $\Delta, \alpha \vdash \tau$, $\rho \in \dpred{\Delta}$, and $R=\vprep{\tau'}$ then
\[
  \vprep{\subst{\tau}{\tau'}{\alpha}} = \vprep[\extsub{\rho}{\alpha}{(\rho_1(\tau'),\rho_2(\tau'),R)}]{\tau} 
\]
\end{lemma}
The lemma says syntactically substituting some type for $\alpha$ into $\tau$ and the interpreting it is the same as semantically substituting the type for $\alpha$. To prove this lemma we would need to show $\vprep{\tau} \in \Rel[\rho_1(\tau),\rho_2(\tau)]$ which is fairly easy given how we have defined our value interpretation.
\begin{proof}[Proof. (Compatibility Lemma 6)]
What we want to show is 
\[
  \inferrule*{\Delta; \Gamma \vdash \lreq{e_1}{e_2} : \forall \alpha . \tau \and
              \Delta \vdash \tau'}
             {\Delta; \Gamma \vdash \lreq{e_1[\tau']}{e_2[\tau']} : \subst{\tau}{\tau'}{\alpha}}
\]
So we assume (1) $\Delta; \Gamma \vdash \lreq{e_1}{e_2} : \forall \alpha . \tau$ and (2) $\Delta \vdash \tau'$.
  \begin{align*}
    & \Delta ; \Gamma \vdash e_1[\tau'] : \subst{\tau}{\tau'}{\alpha} \\
    & \Delta ; \Gamma \vdash e_2[\tau'] : \subst{\tau}{\tau'}{\alpha} \\
    & \forall \rho \in \dpred{\Delta}. \; \forall \gamma \in \gprep{\Gamma}. \; (\rho_1(\gamma_1(e_1[\tau'])),\rho_2(\gamma_2(e_2[\tau']))) \in \eprep{\subst{\tau}{\tau'}{\alpha}}
  \end{align*}
The two first follows from the well-typedness part of (1) together with (2) and the appropriate typing rule. So we only need to show the last one.

Suppose we have a $\rho$ in $\dpred{\Delta}$ and a $\gamma$ in $\gprep{\Gamma}$. We then need to show $(\rho_1(\gamma_1(e_1[\tau'])),\rho_2(\gamma_2(e_2[\tau']))) \in \eprep{\subst{\tau}{\tau'}{\alpha}}$. From the $\curly{E}$-relation we find that to show this we need to show that the two terms run down to two values and those values are related. 

We keep our goal in mind and turn our attention to our premise, $\Delta; \Gamma \vdash \lreq{e_1}{e_2} : \forall \alpha . \tau$. This gives us by definition:
\[
\forall \rho \in \dpred{\Delta}. \; \forall \gamma \in \gprep{\Gamma}. \; (\rho_1(\gamma_1(e_1)),\rho_2(\gamma_2(e_2))) \in \eprep{\tau}
\]
If we instantiate this with the $\rho$ and $\gamma$ we supposed previously, then we get $(\rho_1(\gamma_1(e_1)),\rho_2(\gamma_2(e_2))) \in \eprep{\tau}$ which means that $e_1$ and $e_2$ runs down to some value $v_1$ and $v_2$ where $(v_1,v_2) \in \vprep{\forall \alpha. \; \tau}$. As $(v_1,v_2)$ is in the value interpretation of $\forall \alpha. \; \tau$ we know that the values are of type $\forall \alpha. \; \tau$. From this we know that $v_1$ and $v_2$ are type abstractions, so there must exist $e_1'$ and $e_2'$ such that $v_1 = \tLabs{e_1'}$ and $v_2 = \tLabs{e_2'}$. We can now instantiate $(v_1,v_2) \in \vprep{\forall \alpha. \; \tau}$ with two types and a relation. We use $\rho_1(\tau')$ and $\rho_2(\tau')$ as the two type for instantiation and $\vprep{\tau'}$ as the relation\footnote{Here we use $\vprep{\tau} \in \Rel[\rho_1(\tau),\rho_2(\tau)]$ to justify using the value interpretation as our relation.}. This gives us
\[
  (\subst{e_1'}{\rho_1(\tau')}{\alpha},\subst{e_2'}{\rho_2(\tau')}{\alpha}) \in \eprep[\extsub{\rho}{\alpha}{(\rho_1(\tau'),\rho_2(\tau'),\vprep{\tau'})}]{\tau}
\]
For convenience we write $\rho' = \extsub{\rho}{\alpha}{(\rho_1(\tau'),\rho_2(\tau'),\vprep{\tau'})}$. From the two expressions membership of the expression interpretation we know that $\subst{e_1'}{\rho_1(\tau)}{\alpha}$ and $\subst{e_2}{\rho_2(\tau)}{\alpha}$ run down to some values say $v_{1_f}$ and $v_{2_f}$ respectively where $(v_{1_f},v_{2_f}) \in \vprep[\rho']{\tau}$.

Let us take a step back and see what we have done. We have argued that the following evaluation takes place
\begin{align*}
  \rho_i (\gamma_i(e_i)) [\rho_1(\tau')] & \evaltos (\tLabs{e_i'})[\rho_1(\tau')] \\
                      & \evalto \subst{e_i'}{\rho_1(\tau')}{\alpha} \\
                      & \evaltos v_{i_f}
\end{align*}
where $i=1,2$. The single step in the middle is justified by the type application reduction. The remaining steps are justified in our proof above. If we further note that $\rho_i (\gamma_i (e_i [\tau'])) \equiv  \rho_i (\gamma_i (e_i))[\rho_1(\tau')]$, then we have shown that the two expressions from our goal in fact do run down to two values, and they are related. More precisely we have:
\[
  (v_{1_f},v_{2_f}) \in \vprep[\rho']{\tau}
\]
but that is not exactly what we wanted them to be related under. We are, however, in luck and can apply the compositionality lemma to obtain
\[
  (v_{1_f},v_{2_f}) \in \vprep{\subst{\tau}{\tau'}{\alpha}}
\]
which means that they are related under the relation we needed.
\end{proof}

We call theorems that follows as a consequence of parametricity for free theorems. Next, we will show a free theorem that says that an expression of the type $\forall \alpha. \; \tarrow{\alpha}{\alpha}$ must be the identity function.

\begin{theorem}[Free Theorem (I)] If $\mtenv; \mtenv \vdash e : \forall \alpha.\; \tarrow{\alpha}{\alpha}$, $\mtenv \vdash \tau$, and $\mtenv; \mtenv \vdash v : \tau$, then according to our definition of the logical relation we need to show three things
\[
  e[\tau] \; v \evaltos v
\]
\end{theorem}
System-F is a terminating language, so in the free theorem it suffices to say that when it terminates it is with the value passed as argument. If we had been in a non-terminating language such as System-F with recursive types, then we would have had to state a weaker theorem namely that if the expression terminates, then it is with the value given as argument or the computation diverges.
\begin{proof}
\newcommand{\aaa}{\ensuremath{\forall \alpha. \; \tarrow{\alpha}{\alpha}}}
From the fundamental property and the well-typedness of $e$ we know $\mtenv \vdash \lreq{e}{e} : \aaa$. By definition this gives us
\[
\forall \rho \in \dpred{\Delta}. \; \forall \gamma \in \gprep{\Gamma}. \; (\rho_1(\gamma_1(e)),\rho_2(\gamma_2(e))) \in \eprep{\aaa}
\]
\newcommand{\mt}{\ensuremath{\emptyset}}
We instantiate this with an empty $\rho$ and an empty $\gamma$ to get $(e,e) \in \eprep[\mt]{\aaa}$. From the definition of this we know that $e$ evaluates to some value $F$ and $(F,F) \in \vprep[\mt]{\aaa}$. As $F$ is a value of type \aaa we know the form of $F=\tLabs{e_1}$ for some $e_1$. Now use the fact that $(F,F) \in \vprep[\mt]{\aaa}$ by instantiating it with the type $\tau$ twice and the relation $R=\{(v,v)\}$ to get 
\newcommand{\env}{\ensuremath{\extsub{\mt}{\alpha}{(\tau,\tau,R)}}}
\newcommand{\taa}{\ensuremath{\tarrow{\alpha}{\alpha}}}
$(\subst{e_1}{\tau}{\alpha},\subst{e_1}{\tau}{\alpha})\in \eprep[\env]{\taa}$. We notice that this instantiation is alright as $R \in \Rel[\tau,\tau]$.

The step we just took is an important part a proof of any free theorem namely choosing the relation. Before we chose the relation we picked two types. We did this based on the theorem we want to show. In the theorem we instantiate $e$ with $\tau$, so we pick $\tau$. Likewise with the relation, in the theorem we give $v$ to the function with the domain $\alpha$, so we pick the singleton relation consisting of $(v,v)$. Picking the correct relation is what requires some work in the proof of a theorem. The remaining work done in the proof is simply unfolding of definitions.

Now let us return to the proof. From $(\subst{e_1}{\tau}{\alpha},\subst{e_1}{\tau}{\alpha})\in \eprep[\env]{\taa}$ we know that $\subst{e_1}{\tau}{\alpha}$ evaluates to some value $g$ and $(g,g)\in \vprep[\env]{\taa}$. From the type of $g$ we know that it must be a $\lambda$-abstraction, so $g=\tlabs{x}{\tau}{e_2}$ for some expression $e_2$. Now instantiate $(g,g)\in \vprep[\env]{\taa}$ with $(v,v) \in \vprep[\env]{\alpha}$ to get $(\subst{e_2}{v}{x},\subst{e_2}{v}{x}) \in \eprep[\env]{\alpha}$. From this we know that $\subst{e_2}{v}{x}$ steps to some value $v_f$ and $(v_f,v_f) \in \vprep[\env]{\alpha}$. We have that $\vprep[\env]{\alpha} \equiv R$ so $(v_f,v_f) \in R$ which mean that $v_f = v$ as $(v,v)$ is the only pair in $R$.

Now let us take a step back and consider what we have shown above.
\begin{align*}
  e[\tau] \; v & \evaltos F [\tau] \; v \\
               & \equiv (\tLabs{e_1}) [\tau] \; v \\
               & \evalto (\subst{e_1}{\tau}{\alpha}) \; v \\
               & \evaltos g \; v\\
               & \equiv (\tlabs{x}{\tau}{e_2}) \; v \\
               & \evalto \subst{e_2}{v}{x} \\
               & \evaltos v_f \\
               & \equiv v
\end{align*}
First we argued that $e[\tau]$ steps to some $F$ and that $F$ was a type abstraction, \tLabs{e_1}. Then we performed the type application to get $\subst{e_1}{\tau}{\alpha}$. We then argued that this steps to some $g$ of the form $\tlabs{x}{\tau}{e_2}$ which further allowed us to do a $\beta$-reduction to obtain $\evalto \subst{e_2}{v}{x}$. We then argued that this reduced to $v_f$ which was the same as $v$. In summation we argued $e[\tau] \; v \evaltos v$ which is the result we wanted.
\end{proof}
%We can take a number of steps and end up with v.
%choice of R was critical.
%\forall \alpha. \alpha easier to show (?)
\subsection*{Exercises}
\begin{enumerate}
\item Prove the following free theorem:
  \begin{theorem}[Free Theorem (II)]
    If $\mtenv; \mtenv \vdash e : \forall \alpha. \; (\tarrow{(\tarrow{\tau}{\alpha})}{\alpha})$ and 
       $\mtenv; \mtenv \vdash k : \tarrow{\tau}{\tau_k}$ then
    \[
      \mtenv; \mtenv \vdash \lreq{e[\tau_k] \; k}{k(e[\tau] \; \tlabs{x}{\tau}{x})} : \tau_k
%''\mtenv; \mtenv \vdash'' not in my notes
    \]
  \end{theorem}
  This theorem is a simplified version of the one found in \emph{Theorems For Free} by Philip Wadler. %TODO: insert proper reference (http://ttic.uchicago.edu/~dreyer/course/papers/wadler.pdf)
\end{enumerate}
\clearpage

%\section*{Lecture 5}

\clearpage

\end{document}



