% Dokumentklassen sættes til memoir.
% Manual: http://ctan.org/tex-archive/macros/latex/contrib/memoir/memman.pdf
\documentclass[a4paper,10pt,fleqn]{article}
\usepackage[a4paper]{geometry}
% Danske udtryk (fx figur og tabel) samt dansk orddeling og fonte med

% danske tegn. Hvis LaTeX brokker sig over æ, ø og å skal du udskifte
% "utf8" med "latin1" eller "applemac". 
\usepackage[utf8]{inputenc}
%\usepackage[english]{babel}
\usepackage{csquotes}
\usepackage[T1]{fontenc}
\usepackage{fixltx2e} 
\usepackage{color}
\usepackage{hyperref}

% Matematisk udtryk, fede symboler, theoremer og fancy ting (fx kædebrøker)
\usepackage{amsmath,amssymb,amsfonts}
\usepackage{stmaryrd}
\usepackage{bm}
\usepackage{amsthm}
\usepackage{tikz}
\usetikzlibrary{arrows}
\usetikzlibrary{positioning}
\usetikzlibrary{snakes}
\usetikzlibrary{decorations.pathreplacing}
\usepackage{mathtools}
% Kodelisting. Husk at læse manualen hvis du vil lave fancy ting.
% Manual: http://mirror.ctan.org/macros/latex/contrib/listings/listings.pdf
\usepackage{listings}
\usepackage{verbatim}
\usepackage{enumitem}
\usepackage{bussproofs}
\usepackage{mathpartir}
 
% Fancy ting med enheder og datatabeller. Læs manualen til pakken
% Manual: http://www.ctan.org/tex-archive/macros/latex/contrib/siunitx/siunitx.pdf
%\usepackage{siunitx}
% Indsættelse af grafik.
\usepackage{graphicx}
%\usepackage{bussproofs}

% Reaktionsskemaer. Læs manualen for at se eksempler.
% Manual: http://www.ctan.org/tex-archive/macros/latex/contrib/mhchem/mhchem.pdf
%\usepackage[version=3]{mhchem}

\setlength{\mathindent{1cm}}

\newcommand{\sem}[1]{\ensuremath{\llbracket #1 \rrbracket}}
\newcommand{\tuple}[1]{\ensuremath{\langle #1 \rangle}}
\newcommand{\curly}[1]{\ensuremath{\mathcal{#1}}}

\DeclareMathOperator{\FTV}{FTV}
\DeclareMathOperator{\dom}{dom}
\DeclareMathOperator{\safe}{safe}
\DeclareMathOperator{\irred}{irred}
\DeclareMathOperator{\SNPred}{SN}
\DeclareMathOperator{\val}{Val}
\DeclareMathOperator{\fst}{fst}
\DeclareMathOperator{\snd}{snd}
\DeclareMathOperator{\inl}{inl}
\DeclareMathOperator{\inr}{inr}
\DeclareMathOperator{\scase}{case}
\DeclareMathOperator{\caseof}{of}
\DeclareMathOperator{\fold}{fold}
\DeclareMathOperator{\unfold}{unfold}

\newcommand{\evalto}{\ensuremath{\mapsto}}
\newcommand{\evaltos}[1][*]{\ensuremath{\evalto^{#1}}}
\newcommand{\warning}[1]{{\color{red} !` #1 !} \\}
\newcommand{\mtenv}{{\raisebox{1.5pt}{$\scriptstyle\bullet$}}}

\newcommand{\case}[1]{~\\{\bf Case} #1,~\\}
\newcommand{\eqdef}{\ensuremath{ \; \stackrel{\mathclap{\normalfont\mbox{def}}}{=} \;}}
\newcommand{\subst}[3]{\ensuremath{\ensuremath{#1[#2/#3]}}}
\newcommand{\labs}[2]{\ensuremath{\lambda #1 . \; #2}}
\newcommand{\tlabs}[3]{\ensuremath{\lambda #1 : #2 . \; #3}}
\newcommand{\tLabs}[2][\alpha]{\ensuremath{\Lambda #1 . \; #2}}
\newcommand{\vbar}{\ensuremath{\; | \;}}
\newcommand{\tarrow}[2]{\ensuremath{ #1 \rightarrow #2}}
\newcommand{\eif}[3]{\ensuremath{ \text{if}\; #1 \; \text{then} \; #2 \; \text{else} \; #3}}
\newcommand{\true}{\ensuremath{\text{true}}}
\newcommand{\false}{\ensuremath{\text{false}}}
\newcommand{\SN}[2]{\ensuremath{\SNPred_{#1}(#2)}}
\newcommand{\pand}{\ensuremath{\; \wedge \;}}

\newcommand{\pred}[2]{\ensuremath{\curly{#1}\sem{#2}}}
\newcommand{\pres}[3]{\ensuremath{\curly{#1}_{#2}\sem{#3}}}

\newcommand{\epred}[1]{\ensuremath{\pred{E}{#1}}}
\newcommand{\epres}[2][k]{\ensuremath{\pres{E}{#1}{#2}}}

\newcommand{\vpred}[1]{\ensuremath{\pred{V}{#1}}}
\newcommand{\vpres}[2][k]{\ensuremath{\pres{V}{#1}{#2}}}

\newcommand{\gpred}[1]{\ensuremath{\pred{G}{#1}}}
\newcommand{\gpres}[2][k]{\ensuremath{\pres{G}{#1}{#2}}}

\newcommand{\sub}[3]{\ensuremath{#1[^{#2}/_{#3}]}}

\newcommand*{\circled}[1]{\tikz[baseline=(char.base)]{
            \node[shape=circle,draw,inner sep=2pt] (char) {#1};}}
\newcommand{\TTrue}{\ensuremath{
    \inferrule*[right=T-True]{ }
               {\Gamma \vdash \true : bool}}}

\newcommand{\TFalse}{\ensuremath{
    \inferrule*[right=T-False]{ }
               {\Gamma \vdash \false : bool}}}


\newcommand{\TVar}{\ensuremath{
    \inferrule*[right=T-Var]{\Gamma(x) = \tau}
                            {\Gamma \vdash x : \tau}}}
\newcommand{\TIf}{\ensuremath{
    \inferrule*[right=T-If]{\Gamma \vdash e : bool \and \Gamma \vdash e_1 : \tau \and \Gamma \vdash e_2 : \tau}
               {\Gamma \vdash \eif{e}{e_1}{e_2} : \tau}}}
\newcommand{\TApp}{\ensuremath{
    \inferrule*[right=T-App]{\Gamma \vdash e_1 : \tarrow{\tau_2}{\tau} \and
                            \Gamma \vdash e_2 : \tau_2}
                           {\Gamma \vdash e_1 \; e_2 : \tau}}}

\newcommand{\TAbs}{\ensuremath{\inferrule*[right=T-Abs]{\Gamma, x: \tau \vdash e : \tau_2}
                           {\Gamma \vdash \tlabs{x}{\tau_1}{e} : \tarrow{\tau_1}{\tau_2}}}}
\newcommand{\TFold}{\ensuremath{
    \inferrule*[right=T-Fold]{\Gamma \vdash e : \tau[\mu\alpha. \; \tau/\alpha]}
                             {\Gamma \vdash \fold \; e : \mu\alpha. \; \tau}}}
\newcommand{\TUnfold}{\ensuremath{
    \inferrule*[right=T-Unfold]{\Gamma \vdash e : \mu\alpha. \; \tau}
                               {\Gamma \vdash \unfold \; e : \tau[\mu\alpha. \; \tau/\alpha]}}}
\newcommand{\TFst}{\ensuremath{
    \inferrule*{\Gamma \vdash e : \tau_1 * \tau_2}
               {\Gamma \vdash \fst \; e : \tau_1}}}
\newcommand{\TSnd}{\ensuremath{
    \inferrule*{\Gamma \vdash e : \tau_1 * \tau_2}
               {\Gamma \vdash \snd \; e : \tau_2}}}
\newcommand{\TProd}{\ensuremath{
    \inferrule*{\Gamma \vdash e_1 : \tau_1
                \and
                \Gamma \vdash e_2 : \tau_2}
               {\Gamma \vdash \tuple{e_1,e_2} : \tau_1 * \tau_2}}}


\newcommand{\FTTrue}{\ensuremath{
    \inferrule*[right=T-True]{ \Delta \vdash \Gamma}
               {\Gamma \vdash \true : bool}}}

\newcommand{\FTFalse}{\ensuremath{
    \inferrule*[right=T-False]{\Delta \vdash \Gamma }
               {\Gamma \vdash \false : bool}}}

\newcommand{\FTVar}{\ensuremath{
    \inferrule*[right=T-Var]{\Gamma(x) = \tau \and \Delta \vdash \Gamma}
                            {\Gamma \vdash x : \tau}}}
\newcommand{\FTApp}{\ensuremath{
    \inferrule*[right=T-App]{\Gamma \vdash e_1 : \tarrow{\tau_2}{\tau} \and
                            \Gamma \vdash e_2 : \tau_2 \and 
                            \Delta \vdash \Gamma}
                           {\Gamma \vdash e_1 \; e_2 : \tau}}}

\newcommand{\FTAbs}{\ensuremath{\inferrule*[right=T-Abs]{\Gamma, x: \tau \vdash e : \tau_2 \and \Delta \vdash \Gamma}
                           {\Gamma \vdash \tlabs{x}{\tau_1}{e} : \tarrow{\tau_1}{\tau_2}}}}
\newcommand{\FTIf}{\ensuremath{
    \inferrule*[right=T-If]{\Gamma \vdash e : bool \and 
                            \Gamma \vdash e_1 : \tau \\ 
                            \Gamma \vdash e_2 : \tau \and 
                            \Delta \vdash \Gamma}
               {\Gamma \vdash \eif{e}{e_1}{e_2} : \tau}}}


\newtheorem*{theorem}{Theorem}
\newtheorem*{lemma}{Lemma}
%Lecture 1:
\newtheorem*{strnorm}{Theorem}
\newtheorem*{astrnorm}{Theorem}
\newtheorem*{substlem}{Lemma}
\newtheorem*{forback}{Lemma}
%Lecture 2:
\newtheorem*{stlctypesafety}{Theorem}
\newtheorem*{progress}{Lemma}
\newtheorem*{preservation}{Lemma}
\newtheorem*{btypesafety}{Theorem}
%Lecture 3:
\newtheorem*{stlcmutypesafety}{Theorem}
\newtheorem*{stlcmufundprop}{Theorem}
\newtheorem*{monotonicity}{Lemma}

\author{Lau\\lask@cs.au.dk}
\title{Suggested Solutions for Exercises}
\begin{document}
\maketitle
\section*{Lecture 1 exercises}
\subsection*{1. Prove $\SNPred$ preserved by forward/backward reduction}
The lemma we want to show:
\begin{lemma}
  If $\mtenv \vdash e : \tau$ and $e \evalto e'$ then
  \begin{enumerate}
  \item if $\SN{\tau}{e'}$ then $\SN{\tau}{e}$
  \item if $\SN{\tau}{e}$ then $\SN{\tau}{e'}$
  \end{enumerate}
\end{lemma}
\begin{proof}
  We proof the implications one at a time.
  \begin{enumerate}
  \item Proof by induction on the structure of $\tau$ 
    \case{$\tau=bool$}
    Assume:
    \begin{align}
      &\mtenv \vdash e : bool \label{bass11}\\
      &e\evalto e' \label{bass12} \\
      &\SN{bool}{e'} \label{bass13}
    \end{align}
    Show:
    \[
    \SN{bool}{e} \equiv \mtenv \vdash e : bool \pand e \Downarrow
    \]
    $\mtenv \vdash e : bool$ follows from assumption \ref{bass11}. By definition of assumption \ref{bass13} we have $e' \Downarrow$. Further, from assumption \ref{bass12} we have that we can step from $e$ to $e'$ which gives us $e \Downarrow$
%
%
    \case{$\tau=\tarrow{\tau_1}{\tau_2}$} Assume:
    \begin{align}
      &\mtenv \vdash e : \tarrow{\tau_1}{\tau_2} \label{bass21}\\
      &e\evalto e' \label{bass22} \\
      &\SN{\tarrow{\tau_1}{\tau_2}}{e'} \label{bass23}
    \end{align}
    Show:
    \[
      \SN{\tarrow{\tau_1}{\tau_2}}{e} \equiv \mtenv \vdash e : \tarrow{\tau_1}{\tau_2} \pand e \Downarrow \pand \forall e''. \; \SN{\tau_1}{e''} \implies \SN{\tau_2}{e \; e''}
    \]
    We get $\mtenv \vdash e : \tarrow{\tau_1}{\tau_2}$ and $e \Downarrow$ from an argument similar to the one in the previous case. So it suffices to show
    \[
      \forall e''. \; \SN{\tau_1}{e''} \implies \SN{\tau_2}{e \; e''}
    \]
    Let $e''$ be given and suppose $\SN{\tau_2}{e''}$. From assumption \ref{bass23} we have 
    \[
      \forall e''. \; \SN{\tau_1}{e''} \implies \SN{\tau_2}{e' \; e''}
    \]
    which we instantiate to get $\SN{\tau_2}{e' \; e''}$. 
    Now consider one of the induction hypotheses:
    \[
    \forall e_1 e_2.\; \mtenv \vdash e_1 : \tau_2 \pand e_1 \evalto e_2 \pand \SN{\tau_2}{e_2} \implies \SN{\tau_2}{e_1}
    \]
    We want to instantiate this with $e_1 = e \; e''$ and $e_2 = e' \; e''$. We get $\mtenv \vdash e \; e'' : \tau_2$ from assumption \ref{bass21} and $\SN{\tau_2}{e''}$ along with the typing rule for application.
    \begin{comment}
      \[
      \inferrule*{\inferrule*{ (\ref{bass21}) }
        {\mtenv \vdash e : \tarrow{\tau_1}{\tau_2}}
        \and
        \inferrule*{ \SN{\tau_2}{e''} }
                   {\mtenv \vdash e'' : \tau_1}}
                 {\mtenv \vdash e \; e'' : \tau_2}
                 \]
    \end{comment}
To get $e \; e'' \evalto e' \; e''$ we use assumption \ref{bass22} along with our evaluation rules (use the evaluation context $[]\; e''$). Finally, we got $\SN{\tau_2}{e' \; e''}$ just before we considered the induction hypothesis above. We therefore got everything needed to instantiate the induction hypothesis which gives us $\SN{\tau_2}{e' \; e''}$, which was what we needed to be done.
  \item Proof by induction on the structure of $\tau$
    \case{$\tau=bool$} 
Assume
    \begin{align}
      &\mtenv \vdash e : bool \label{bass31}\\
      &e\evalto e' \label{bass32} \\
      &\SN{bool}{e} \label{bass33}
    \end{align}
    Show:
    \[
    \SN{bool}{e'} \equiv \mtenv \vdash e' : bool \pand e' \Downarrow
    \]
  We get $e' \Downarrow$ from the assumptions \ref{bass32} and \ref{bass33} and the fact that our language is deterministic. From \ref{bass33} we specifically use $e \Downarrow$. If we know that $e$ can take several steps to some value, and we know that the first step $e$ takes is to $e'$, then $e'$ will step to the same value. 
To argue $\mtenv \vdash e' : bool$ we bring out the big guns, namely the preservation lemma\footnote{Here I will just assume the preservation lemma, as the full proof is rather extensive. It is however a well-known lemma for the simply typed lambda calculus, and a good account of the proof can be found in Benjamin Pierce's \emph{Types and Programming Languages.}}:
\[
  \Gamma \vdash e : \tau \pand e \evalto e' \implies \Gamma \vdash e' : \tau
\]
If we use this lemma with assumption \ref{bass31} and \ref{bass32} we get $\mtenv \vdash e' : bool$.
%
%
    \case{$\tau=\tarrow{\tau_1}{\tau_2}$} Assume:
    \begin{align}
      &\mtenv \vdash e : \tarrow{\tau_1}{\tau_2} \label{bass41}\\
      &e\evalto e' \label{bass42} \\
      &\SN{\tarrow{\tau_1}{\tau_2}}{e} \label{bass43}
    \end{align}
    Show:
    \[
    \SN{\tarrow{\tau_1}{\tau_2}}{e'} \equiv \mtenv \vdash e' : \tarrow{\tau_1}{\tau_2} \pand e' \Downarrow \pand \forall e''. \; \SN{\tau_1}{e''} \implies \SN{\tau_2}{e' \; e''}
    \]
    The argument for $\vdash e' : \tarrow{\tau_1}{\tau_2}$ and $e' \Downarrow$ is similar to the one we did in the $\tau=bool$ case. So it suffices to show:
    \[
    \SN{\tau_1}{e''} \implies \SN{\tau_2}{e' \; e''}
    \]
    The reasoning here is similar to the same reasoning done in the same situation for the backwards part of the lemma. In short it was: suppose $\SN{\tau_1}{e''}$ have $\mtenv \vdash e \; e'' : \tau_2$ from assumption \ref{bass41}, $\SN{\tau_1}{e''}$ and the application typing rule. Moreover, have $e \; e'' \evalto e' \; e''$ from assumption \ref{bass42} and the evaluation rules. Use these three things to instantiate the induction hypothesis:
\[
  \forall e_1, e_2.\; \mtenv \vdash e_1 : \tau_2 \pand e_1 \evalto e_2 \pand \SN{\tarrow{\tau_1}{\tau_2}}{e_1} \implies \SN{\tau_2}{e_2}
\]
to get \SN{\tau_2}{e' \; e''}.
  \end{enumerate}
\end{proof}

\subsection*{2. Prove the substitution lemma}
%TODO


\subsection*{3. Prove the if-case of ``\circled{a} Generaliesd''}
Prove the if case of
\begin{theorem}
  If $\Gamma \vdash e : \tau$ and $\gamma \models \Gamma$, then $\SN{\tau}{\gamma(e)}$
\end{theorem}
\begin{proof}The proof is by induction on the typing derivation.
  \case{$\TIf$}
  Assume:
  \begin{align}
    & \Gamma \vdash \eif{e}{e_1}{e_2} : \tau \\
    & \gamma \models \Gamma
  \end{align}
  Show:
  \[
  \SN{\tau}{\gamma(\eif{e}{e_1}{e_2})} \equiv   \SN{\tau}{\eif{\gamma(e)}{\gamma(e_1)}{\gamma(e_2)}}
  \]
  In this case we have the following induction hypotheses:
  \begin{align*}
    & \Gamma \vdash e : bool \pand \gamma \models \Gamma \implies \SN{bool}{\gamma(e)} \\
    & \Gamma \vdash e_1 : \tau \pand \gamma \models \Gamma \implies \SN{\tau}{\gamma(e_1)} \\
    & \Gamma \vdash e_2 : \tau \pand \gamma \models \Gamma \implies \SN{\tau}{\gamma(e_2)}
  \end{align*}
\end{proof}
We can immediately apply the induction hypotheses as we have that $e$, $e_1$, and $e_2$ are well-typed with respect to $bool$, $\tau$, and $tau$, repspectively aswell as $\gamma \models \Gamma$. We thus have $\SN{bool}{\gamma(e)}$, $\SN{\tau}{\gamma(e_1)}$, and $\SN{\tau}{\gamma(e_2)}$.

If we can show $\eif{\gamma(e)}{\gamma(e_1)}{\gamma(e_2)}$ is closed and well-typed and steps to $\gamma(e_1)$ or $\gamma(e_2)$. We can then case on what the if-expression steps to and in both cases apply the backwards part of the $\SNPred$ preservation lemma $\SN{\tau}{\eif{\gamma(e)}{\gamma(e_1)}{\gamma(e_2)}}$ which is the result we need.

So it suffices to show $\eif{\gamma(e)}{\gamma(e_1)}{\gamma(e_2)} \evaltos \gamma(e_1)$ or $\eif{\gamma(e)}{\gamma(e_1)}{\gamma(e_2)} \evaltos \gamma(e_1)$ and $\mtenv \vdash \eif{\gamma(e)}{\gamma(e_1)}{\gamma(e_2)} : \tau$. 

Let us first argue that the if-expression is well-typed and closed: from $\SN{bool}{\gamma(e)}$ we get $\mtenv \vdash \gamma(e) : bool$, from $\SN{\tau}{\gamma(e_1)}$ we get $\mtenv \vdash \gamma(e_1) : \tau$, and from $\SN{\tau}{\gamma(e_2)}$ we get $\mtenv \vdash \gamma(e_2) : \tau$. This can be used with the typing rule for if-expressions to conclude $\mtenv \vdash \eif{\gamma(e)}{\gamma(e_1)}{\gamma(e_2)} : \tau$.

From $\SN{bool}{\gamma(e)}$ we know that $\gamma(e) \Downarrow b$, for some value $b$. We further know $\SN{bool}{b}$ from the forward part of the $\SNPred$ preservation lemma\footnote{The lemma can be applied successively until $b$ is reached.}. This gives us $\mtenv \vdash b : bool$ which we can use with the canonical forms lemma\footnote{Not proven here. See Benjamin Pierce's \emph{Types and Programming Languages} for an account of how to prove it.} to get $b=\true$ or $b=\false$. If $b=\true$, then by the evaluation rule for if-expressions we have $\eif{\true}{\gamma(e_1)}{\gamma(e_2)} \evalto \gamma(e_1)$, so we have demonstrated that $\eif{\gamma(e)}{\gamma(e_1)}{\gamma(e_2)} \evaltos \gamma(e_1)$. Similar for $b = \false$ we get $\eif{\gamma(e)}{\gamma(e_1)}{\gamma(e_2)} \evaltos \gamma(e_2)$.

We have now shown all it sufficed to show, so we are done.

\subsection*{4. Extend the language with pairs and do the proofs}
First we extend the language with pairs as follows:
\begin{align*}
  \tau & ::= \dots \vbar \tau * \tau \\
  e    & ::= \dots \vbar \tuple{e,e} \vbar \snd \; e \vbar \fst \; e \\
  v    & ::= \dots \vbar \tuple{v,v} \\
  E    & ::= \dots \vbar \fst \; E \vbar \snd \; E \vbar \tuple{E,e} \vbar \tuple{v,E} \\
\end{align*}
\begin{align*}
  \fst \; \tuple{v_1,v_2} \evalto v_1 \\
  \snd \; \tuple{v_1,v_2} \evalto v_2 
\end{align*}
\[
\TFst
\hspace{1cm}
\TSnd
\hspace{1cm}
\TProd
\]
Finally we extend the logical predicate:
\[
\SN{\tau_1*\tau_2}{e} \iff \mtenv \vdash e : \tau_1 * \tau_2 
                        \pand e \Downarrow
                        \pand \SN{\tau_1}{\fst \; e}
                        \pand \SN{\tau_2}{\snd \; e}
\]
Now we just need to adjust all our proofs. We need to show safety which we did in two parts, \circled{a} and \circled{b}. Furthermore, we had a substitution lemma and a $\SNPred$ preserved by backward and forward reduction lemma.
\subsubsection*{Type safety, \circled{b}}
The \circled{b} part of type safety states that if an expression $e$ is in the interpretation of $\tau$, then it is strongly normalising. This is indeed the case after extended $\SNPred$, as it is one of the conditions of the predicate. So this result follows trivially.


\subsubsection*{Substitution lemma}
%TODO


\subsubsection*{$\SNPred$ preserved by forward reduction}
\begin{lemma}
  If $\mtenv \vdash e : \tau$, $e \evalto e'$, and $\SN{\tau}{e}$, then $\SN{\tau}{e'}$
\end{lemma}
\begin{proof}
  The proof is by induction on the structure of $\tau$. The cases we had in the proof before we added products does not change, so it only remains to show the product case:
  \case{$\tau = \tau_1 * \tau_2$}
  Assume:
  \begin{align*}
    & \mtenv \vdash e : \tau_1 * \tau_2 \\
    & e \evalto e' \\
    & \SN{\tau_1*\tau_2}{e} \equiv \mtenv \vdash e : \tau_1 * \tau_2  \pand e \Downarrow \pand \SN{\tau_1}{\fst \; e} \pand \SN{\tau_2}{\snd \; e}
  \end{align*}
  Show:  
  \begin{align}
    \SN{\tau_1*\tau_2}{e'} \equiv & \nonumber \\
                                &\mtenv \vdash e' : \tau_1 * \tau_2  \pand \label{snf:1}\\
                                &e' \Downarrow \pand \label{snf:2}\\
                                &\SN{\tau_1}{\fst \; e'} \pand \label{snf:3} \\
                                &\SN{\tau_2}{\snd \; e'} \label{snf:4}
  \end{align}
  \ref{snf:1} follows from the preservation lemma\footnote{We do not show the preservation lemma here.} used with $\mtenv \vdash e : \tau_1 * \tau_2$ and $e \evalto e'$. \ref{snf:2} follows from $e \evalto e'$ and $e \Downarrow$. This time the argument is that our evaluation rules are deterministic, so if $e$ steps to $e'$ and $e$ is strongly normalising, then $e'$ will also be strongly normalising (and it will evaluate to the same value as $e$). Finally, we show \ref{snf:3} and \ref{snf:4} in similar fashion, so we will only show \ref{snf:3}. To do this we apply one of the induction hypotheses, namely:
\[
  \mtenv \vdash \fst e : \tau_1 \pand \fst \; e \evalto \fst \; e' \pand \SN{\tau_1}{\fst \; e} \implies \SN{\tau_1}{\fst e'}
\]
$\mtenv \vdash \fst e : \tau_1$ follows from $\mtenv \vdash e : \tau_1 * \tau_2$ and the appropriate typing rule. $\fst \; e \evalto \fst \; e'$ follows from the evaluation rule about evaluation contexts, $e \evalto e'$ and the context $E = \fst\; []$. Finally, we have $\SN{\tau_1}{\fst \; e}$ from one of our initial assumptions. We now have all the premises of the induction hypothesis, so we apply it to get our desired result, namely $\SN{\tau_1}{\fst e'}$.u
\end{proof}


\subsubsection*{$\SNPred$ preserved by backward reduction}
\begin{lemma}
  If $\mtenv \vdash e : \tau$, $e \evalto e'$, and $\SN{\tau}{e'}$, then $\SN{\tau}{e}$
\end{lemma}
\begin{proof}
  The proof is by induction on the structure of $\tau$. The cases we had in the proof before we added products does not change, so it only remains to show the product case:
  \case{$\tau = \tau_1 * \tau_2$}
  Assume:
  \begin{align*}
    & \mtenv \vdash e : \tau_1 * \tau_2 \\
    & e \evalto e' \\
    & \SN{\tau_1*\tau_2}{e'} \equiv \mtenv \vdash e' : \tau_1 * \tau_2  \pand e' \Downarrow \pand \SN{\tau_1}{\fst \; e'} \pand \SN{\tau_2}{\snd \; e'}
  \end{align*}
  Show:  
  \begin{align}
    \SN{\tau_1*\tau_2}{e} \equiv & \nonumber \\
                                &\mtenv \vdash e : \tau_1 * \tau_2  \pand \label{snb:1}\\
                                &e \Downarrow \pand \label{snb:2}\\
                                &\SN{\tau_1}{\fst \; e} \pand \label{snb:3} \\
                                &\SN{\tau_2}{\snd \; e} \label{snb:4}
  \end{align}
  We need to show the four condition above. \ref{snb:1} follows directly from the assumptions. \ref{snb:2} follows from $e \evalto e'$ and $e' \Downarrow$. We can first take a step from $e$ to $e'$ and $e'$ is strongly normalising, thus so is $e$. \ref{snb:3} and \ref{snb:4} are shown in similarly, so we just show \ref{snb:3} here. To show this we use the induction hypothesis that says:
\[
  \mtenv \vdash \fst e : \tau_1 \pand \fst \; e \evalto \fst \; e' \pand \SN{\tau_1}{\fst \; e'} \implies \SN{\tau_1}{\fst e}
\]
We get $\mtenv \vdash e : \tau_1$ from the approriate typing rule and $\mtenv \vdash e : \tau_1 * \tau_2$. $\fst \; e \evalto \fst \; e'$ follows from $e \evalto e'$ and the evaluation rules if we use the context $E=\fst \; []$. Finally, $\SN{\tau_1}{\fst \; e'}$ was assumed intially. We can now apply the induction hypothesis which gives us the desired result, $\SN{\tau_1}{\fst \; e}$.
\end{proof}


\subsubsection*{Type safety, \circled{a}}
To show \circled{a} we show the generalised version:
\begin{theorem}
  If $\Gamma \vdash e : \tau$ and $\gamma \models \Gamma$, then $\SN{\tau}{\gamma(e)}$
\end{theorem}
\begin{proof}
  The proof is by induction on the typing derivation. The cases, we had before adding products, does not change, so it only remains to show the new cases.
  \case{\TFst}
  Assume:
  \begin{align*}
    & \gamma \models \Gamma \\
    & \Gamma \vdash e : \tau_1 * \tau_2
  \end{align*}
  Show:
  \[
  \SN{\tau_1}{\gamma(\fst \; e)} \equiv \SN{\tau_1}{\fst \; \gamma(e)}
  \]
  To show this we first apply the induction hypothesis:
  \[
    \gamma \vdash e : \tau_1 * \tau_2 \pand \gamma \models \Gamma \implies \SN{\tau_1*\tau_2}{\gamma(e)}
  \]
  which we can do immediately using our assumptions. From $\SN{\tau_1*\tau_2}{\gamma(e)}$ we get that $\gamma(e)$ is strongly normalising, so there exists a value $v$ that $\gamma(e)$ evaluates to, i.e., $\gamma(e) \evaltos v$. Using this along with $\SN{\tau_1*\tau_2}{\gamma(e)}$ and the forward part of the $\SNPred$ preservation lemma\footnote{We may need to apply it repeatedly.} we can now conclude $\SN{\tau_1*\tau_2}{v}$. By definition of $\SNPred$ this further gives us $\SN{\tau_1}{\fst \; v}$. We can further use $\gamma(e) \evaltos v$ to conclude $\fst \; \gamma(e) \evaltos \fst \; v$ which is done using the evaluation rule repeatedly in the evaluation context $E=\fst \; []$. Assume for now that we have the well-typedness conditions we need to use the backwards part of the $\SNPred$ preservation lemma with $\fst \; \gamma(e) \evaltos \fst \; v$ and $\SN{\tau_1}{\fst \; v}$. If we have that then we get the result we want, $\SN{\tau_1}{\fst \; \gamma(e)}$. So it suffices to argue the well-typedness condition is okay all the way from $\fst \; v$ up to $\fst \; \gamma(e)$. We get this by first using the substitution lemma with $\gamma \vdash e : \tau_1 * \tau_2$ and $\gamma \models \Gamma$ to get $\mtenv \vdash \gamma(e) : \tau_1 * \tau_2$. This we can use with the typing rule for $\fst$ to get $\mtenv \vdash \fst \; \gamma(e) : \tau_1$. We can then use preservation to argue that all the intermediate expressions when evaluating from $\fst \gamma(e)$ to $\fst \; v$ are all well typed.
  \case{\TSnd} This case is symmetric to the one for $\fst$, so it is omitted here.
  \case{\TProd} 
  Assume:
  \begin{align*}
    & \gamma \models \Gamma \\
    & \Gamma \vdash e_1 : \tau_1 \\
    & \Gamma \vdash e_2 : \tau_2
  \end{align*}
  Show:
  \begin{align}
    \SN{\tau_1*\tau_2}{\gamma(\tuple{e_1,e_2})} \equiv & \SN{\tau_1*\tau_2}{\gamma(\tuple{\gamma(e_1),\gamma(e_2)})} \equiv \nonumber \\
                                              & \mtenv \vdash \tuple{\gamma(e_1),\gamma(e_2)} : \tau_1 * \tau_2 \label{tsb:1} \pand \\
                                              & \tuple{\gamma(e_1),\gamma(e_2)} \Downarrow \label{tsb:2} \pand \\
                                              & \SN{\tau_1}{\fst \; \tuple{\gamma(e_1),\gamma(e_2)}} \label{tsb:3} \pand\\
                                              & \SN{\tau_2}{\snd \; \tuple{\gamma(e_1),\gamma(e_2)}} \label{tsb:4}    
  \end{align}
  We can show \ref{tsb:1} by first applying the substitution lemma twice with $\gamma \models \Gamma$ as the substitution in both cases and with $\Gamma \vdash e_1 : \tau_1$ and $\Gamma \vdash e_2 : \tau_2$ as the two typing judgements. This gives us $\Gamma \vdash \gamma(e_1) : \tau_1$ and $\Gamma \vdash \gamma(e_2) : \tau_2$. It is then a simple matter of applying the typing rule to get $\Gamma \vdash \tuple{e_1,e_2}: \tau_1 * \tau_2$.

To show the rest we need to use the induction hypotheses:
  \begin{align*}
   & \Gamma \models \gamma \pand \Gamma \vdash e_1 : \tau_1 \implies \SN{\tau_1}{\gamma(e_1)} \\
   & \Gamma \models \gamma \pand \Gamma \vdash e_2 : \tau_2 \implies \SN{\tau_2}{\gamma(e_2)}
  \end{align*}
  We can apply the induction hypotheses immidiately to get $\SN{\tau_1}{\gamma(e_1)}$ and $\SN{\tau_2}{\gamma(e_2)}$. No matter what type $\tau_1$ and $\tau_2$ are we have the following:
  \begin{align*}
   &\SN{\tau_1}{\gamma(e_1)} \implies \gamma(e_1) \Downarrow \\
   &\SN{\tau_2}{\gamma(e_2)} \implies \gamma(e_2) \Downarrow
  \end{align*}
  Assume that $\gamma(e_1)$ and $\gamma(e_2)$ respectively evaluate to some values $v_1$ and $v_2$. If we inspect the evaluation rules we see that this gives us
  \[
    \tuple{\gamma(e_1),\gamma(e_2)} \evaltos \tuple{v_1,\gamma(e_2)} \evaltos \tuple{v_1,v_2}
  \]
So we have shown \ref{tsb:2}. It remains to show \ref{tsb:3} and \ref{tsb:4}. The two turns out to be symmetric, so we will just show \ref{tsb:3} here. To do so we first observe that the above evaluation used with the evaluation rules in the evaluation context $E=\fst \; []$ gives us
  \[
    \fst \; \tuple{\gamma(e_1),\gamma(e_2)} \evaltos v_1
  \]
We further had $\SN{\tau_1}{\gamma(e_1)}$ from our induction hypothesis which we use with $\gamma(e_1) \evaltos v_1$ (follows from the strong normalisation property of $\SN{\tau_1}{\gamma(e_1)}$) and $\mtenv \vdash \gamma(e_1) : \tau_1$ (which we had previously as an intermediate derivation) with the $\SNPred$ forward preservation lemma to conclude $\SN{\tau_1}{v_1}$. If we use $\mtenv \vdash \tuple{\gamma(e_1),\gamma(e_2)}: \tau_1 * \tau_2$ with the $\fst$ typing rule, we get $\mtenv \vdash \fst \; \tuple{\gamma(e_1),\gamma(e_2)}: \tau_1 * \tau_2$ which we can use with the preservation lemma and the evaluation to $v_1$ to conclude that every intermediate step is closed and well-typed. This sets the scene for the backward part of the $\SNPred$ preservation lemma which we use repeatedly to conclude $\SN{\tau_1}{\fst \; \tuple{\gamma(e_1),\gamma(e_2)}}$ which is the result we wanted.
\end{proof}

\section*{Lecture 2 exercies}
\subsection*{Prove the TApp case of the Fundemental Property theorem}
%TODO


\section*{Lecture 3 exercises}
\subsection*{2. Attempt to prove monotonicity with adjusted value interpretation}
Exercise:
\begin{displayquote}
Try to prove the monotonicity lemma where the definition of the value interpretation has been adjusted with:
\[
\vpres{\tarrow{\tau_1}{\tau_2}} = \{\tlabs{x}{\tau_1}{e} \vbar \; \forall v \in \vpres{\tau_1}. \; \sub{e}{v}{x} \in \epres{\tau_2} \}
\]
This will fail, but it is instructive to see how it fails.
\end{displayquote}
\begin{theorem}
  If $v\in \vpres{\tau}$ and $j \leq k$  then $v \in \vpres[j]{\tau}$.
\end{theorem}
\begin{proof}\renewcommand{\qedsymbol}{$\times$}
We attempt a proof by induction over the structure of $\tau$.
\case{$\tau = \tarrow{\tau_1}{\tau_2}$} 
We assume 
\begin{align}
  & v \in \vpres{\tau} \label{eq1} \\
  & j \leq k           \nonumber 
\end{align}
We need to show 
\begin{equation}
  v\in\vpres[j]{\tau}    \label{eq3}
\end{equation}
To show \ref{eq3} we suppose $v\in \vpres[j]{\tau_1}$, and then it suffices to show $\subst{e}{v}{x} \in \epres[j]{\tau_2}$. By definition we know from \ref{eq1} 
\begin{equation}
  \forall v \in \vpres{\tau_1}.\; \subst{e}{v}{x} \in \epres{\tau_2} \label{eq4}
\end{equation}
Now let us take a step back and consider what we got, and what we want to show. We want to show $\subst{e}{v}{x} \in \epres[j]{\tau_2}$, but the only thing we got that mentions the expression intepretation is \ref{eq4}, but it only talks about the expression interpretation at step $k$, and we need it at step $j$. Now let us consider the induction hypotheses:
\begin{equation}
  v\in \vpres{\tau_n} \wedge 
  j \leq k 
  \implies v \in \vpres[j]{\tau_n}\quad, n=1,2 \nonumber
\end{equation}
We do have (\ref{eq1}) that talks about the value interpretation for $\tau$, but we already unfolded that and it did not give us anything that talks about the value interpretation for $\tau_1$ or $\tau_2$ at $k$ steps. It would seem like we are stuck.
\end{proof}


\section*{Lecture 4 exercises}
\subsection*{Prove the followinf free theorem}  
\begin{theorem}[Free Theorem (II)]
    If $\mtenv; \mtenv \vdash e : \forall \alpha. \; (\tarrow{(\tarrow{\tau}{\alpha})}{\alpha})$ and 
       $\mtenv; \mtenv \vdash k : \tarrow{\tau}{\tau_k}$ then
    \[
      \mtenv; \mtenv \vdash \leq{e[\tau_k] \; k}{k(e[\tau] \; \tlabs{x}{\tau}{x})} : \tau_k
%''\mtenv; \mtenv \vdash'' not in my notes
    \]
  \end{theorem}
%TODO


\end{document}
