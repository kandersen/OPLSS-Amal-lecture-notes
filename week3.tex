\section*{Universal Types and Relational Substitutions}
In the previous sections we considered safety and termination, but now we shift our focus to program equivalence. To prove program equivalences we will need logical relations. To motivate the need for arguing about program equivalence we first introduce universal types.

Say we have a function that sorts integer lists:
\[
  sortint \; : \; \tarrow{list \; int}{list \; int}
\]
$sortint$ takes a list of integers and returns a sorted version of that list. Say we now want a function that sorts lists of strings, then instead of implementing a separate function we could factor out the code responsible for sorting and have just one function. The type signature of such a generic sort function is:
\[
  sort \; : \; \forall \alpha.\; \tarrow{(list \; \alpha) \times (\tarrow{\alpha \times \alpha}{bool} )}{list \; \alpha}
\]
$sort$ takes a type, a list of elements of this type, and a comparison function that compares to elements of the type argument and returns a list sorted according to the comparison function. An example of an application of this function could be
\[
  sort \; [int] \; (3,7,5) \; <
\]
Whereas sort instantiated with the $string$ type, but given an integer list would not be a well typed instantiation.
\[
  sort \; [string] \; \cancelto{("a","c","b")}{(3,7,5)} \; \text{string<}
\]
Here the application with the list $(3,7,5)$ is not well typed, but if we instead use a list of strings, then it type checks. 

We want to extend the simply typed lambda calculus with functions that abstract over types in the same way lambda abstractions, $\tlabs{x}{\tau}{e}$, abstract over terms. We do that by introducing a type abstraction:
\[
  \tLabs{e}
\]
This function abstracts over the type $\alpha$ which allows $e$ to depend on $\alpha$.

\subsection*{System F (STLC with universal types)}
\begin{align*}
  \tau &::= \dots \vbar \forall \alpha. \; \tau \\
  e    &::= \dots \vbar \tLabs{e} \vbar e[\tau] \\
  v    &::= \dots \vbar \tLabs{e}\\
  E    &::= \dots \vbar E[\tau] 
\end{align*}
\[
  (\tLabs{e})[\tau] \evalto e[\tau/\alpha]
\]
Type environment\footnote{We do not annotate $\alpha$ with a kind, as we only have one kind in this language.} (The type environment is assumed to consist of distinct type variables. For instance, the environment $\Delta,\alpha$ is only well formed if $\alpha \not\in \dom(\Delta)$).
\[
  \Delta ::= \mtenv \vbar \Delta,\; \alpha
\]
With the addition of type environments of type variables our typing judgments now have the following form:
\[
  \Delta,\Gamma \vdash e : \tau
\]
We now need a notion of well-formed types. If $\tau$ is well formed with respect to $\Delta$, then we write:
\[
  \Delta \vdash \tau
\]
We do not include the formal rules here, but they amount to $\FTV(\tau) \subseteq \Delta$, where $\FTV(\tau)$ is the set of free type variables in $\tau$.

We further introduce a notion of well formed environments. An environment is well formed if all the types that appear in the range of $\Gamma$ are well formed.
\[
  \Delta \vdash \Gamma \eqdef \forall x \in \dom(\Gamma). \; \Delta \vdash \Gamma(x)
\]
For any type judgment $\Delta,\Gamma \vdash e : \tau$ we have as an invariant that $\tau$ is well-formed in $\Delta$ and $\Gamma$ is well formed in $\Delta$. The old typing system modified to use the new form of the typing judgment looks like this:
\[
  \FTFalse
\hspace{1cm}
  \FTTrue
\]
\[
  \FTVar
\hspace{1cm}
  \FTIf
\]  
\[
  \FTAbs 
\hspace{1cm}
  \FTApp
\]
Notice that the only thing that has changed is that $\Delta$ has been added to the environment in the judgments. We further extend the typing rules with the following two rules to account for our new language constructs:
\[
  \FTtAbs
\hspace{1cm}
  \FTtApp
\]
\subsubsection*{Properties of System-F}
In System-F certain types reveal the behavior of the functions with that type. Let us consider terms with the type $\forall \alpha.\; \tarrow{\alpha}{\alpha}$. Recall from the motivation in lecture 1 that this had to be the identity function. We can now phrase this as a theorem:
\begin{theorem}
  If $\mtenv ; \mtenv \vdash e : \forall \alpha. \; \tarrow{\alpha}{\alpha}$,\\ $\mtenv \vdash \tau$, and\\ $\mtenv; \mtenv \vdash v : \tau$,\\ then $e[\tau]\; v \evaltos v$
\end{theorem}
This is a free theorem in this language. Another free theorem that was mentioned in the motivation of lecture 1 was about expressions with type $\forall \alpha. \tarrow{\alpha}{bool}$. Here all expressions with this type had to be constant functions. We can also phrase this as a theorem
\begin{theorem}
  If $\mtenv \vdash \tau$, $\mtenv \vdash v_1 : \tau$,\\ and $\mtenv \vdash v_1 : \tau$,\\ then $\ctxeq{e[\tau] \; v_1}{e[\tau]\; v_2}$.
\end{theorem}
Or in a slightly more general fashion where we allow different types:
\begin{theorem}
  If $\mtenv \vdash \tau$, \\ $\mtenv \vdash \tau'$, \\$\mtenv \vdash v_1 : \tau$,\\ and $\mtenv \vdash v_1 : \tau'$,\\ then $\ctxeq{e[\tau]\; v_1}{e[\tau']\; v_2}$.
\end{theorem}
\footnote{We have not yet defined $\ctxeq{}{}$ so for now just treat it as the two programs are equivalent without thinking too much about what equivalence means.}We get these free theorems because the functions have no way of inspecting the argument as they do not know what type it is. As the function has to treat its argument as an unknown ``blob'' it has no choice but to return the same value every time.

The question now is: ``how do we prove these free theorems?'' The two last theorems both talk about program equivalence which we prove using logical relations. The first theorem did not mention equivalence, but the proof technique of choice is still a logical relation.

\subsection*{Contextual Equivalence}
To define a contextual equivalence we first define the notion of a program context. A program context is a complete program with exactly one hole in it. It is defined as follows:
\begin{align*}
  C ::= &\;  [\cdot] \\
        &\vbar \eif{C}{e}{e}  \\
        &\vbar \eif{e}{C}{e} \\
        &\vbar \eif{e}{e}{C} \\
        &\vbar \tlabs{x}{\tau}{C} \\
        &\vbar C \; e \\
        &\vbar e \; C \\
        &\vbar \tLabs{C} \\
        &\vbar C[\tau]
\end{align*}
We need a notion of context typing. For simplicity we just introduce it for simply typed lambda calculus. The context typing is written as:
%TODO: Hmm, some way of making this clearer? Natural language or perhaps
%an inference rule of some sort?
\[
C:\; (\Gamma \vdash \tau) \implies (\Gamma' \vdash \tau') \implies \Gamma \vdash e : \tau \implies \Gamma' \vdash C[e] : \tau'
\]
This means that for any expression $e$ of type $\tau$ under $\Gamma$ if we embed it into $C$, then the type of the embedding is $\tau'$ under $\Gamma'$.

Informally we want contextual equivalence to say that no matter what program context we embed either of the two expressions in, it gives the same result. This is also called as observational equivalence as the program context is unable to observe any difference no matter what expression we embed in it. We can of course not plug an arbitrary term into the hole, so we annotate the equivalence with the type of the hole which means that the two contextual equivalent expressions have to have that type.
\[
  \Delta; \Gamma \vdash \ctxeq{e_1}{e_2} : \tau \eqdef \forall C.\; (\Delta ; \Gamma \vdash \tau ) \implies (\mtenv ; \mtenv \vdash \tau') \implies (C[e_1] \Downarrow v \iff C[e_2] \Downarrow v)
\]
This definition assumes that $e_1$ and $e_2$ has type $\tau$ under the specified contexts.

Contextual equivalence is handy because we want to be able to reason about the equivalence of two implementations. Say we have two implementations of a stack, one is implemented using an array the other using a list. If we can show that the two implementations are contextual equivalent, then we can use the more efficient one over the less efficient one and know that the complete program will behave the same.
%TODO: perhaps also a word on how we could take the simpler of the two
      %as ``specification'', and hence show correctness of a more
      %efficient, fast or otherwise clever implementation _with
      %respect to_ the specification?

In the next lecture we will introduce a logical relation such that
\[
  \Delta ; \Gamma \vdash \lreq{e_1}{e_2} : \tau \implies \text{contextual equivalence \ctxeq{}{}} 
\]
That is we want to show that the logical relation is sound with respect to contextual equivalence. 

If we can prove the above soundness, then we can state our free theorems with $\lreq{}{}$ rather than $\ctxeq{}{}$ and get the same result if we can prove the logical equivalence. We would like to do this as it is difficult to directly prove two things are contextual equivalent. A direct proof has to talk about all possible program contexts which we could do using induction, but the lambda-abstraction case turns out to be difficult. This motivates the use of other proof methods where using a logical relation is one of them.

\subsection*{A Logical Relation for System F}
Now we need to build a logical relation for System F. With this logical relation we would like to be able to prove the free theorems from lecture 3. Our value interpretation will now consist of pairs as we are defining a relation. The value relation will have the following form:
\[
  \vpred{\tau} = \{(v_1,v_2) \vbar 
                        \mtenv ; \mtenv \vdash v_1 : \tau \pand 
                        \mtenv ; \mtenv \vdash v_1 : \tau \pand \dots \}
\]
In our value interpretation we require $v_1$ and $v_2$ to be closed and well typed, but for succinctness we do not write this in the definitions below. 
Let us try to naively build the logical relation the same way we build the logical predicates:
\begin{align*}
  \vpred{bool}                 & = \{ (\true,\true), (\false,\false) \} \\
  \vpred{\tarrow{\tau}{\tau'}} & = \{ (\tlabs{x}{\tau}{e_1},\tlabs{x}{\tau}{e_2}) \vbar \forall (v_1,v_2) \in \vpred{\tau}. \; (\subst{e_1}{v_1}{x},\subst{e_2}{v_2}{x}) \in \epred{\tau'} \}
\end{align*}
The value interpretation of the function type is defined based on the slogan for logical relations ``related inputs to related outputs.'' If we had chosen to use equal inputs rather than related, then our definition would be more restrictive than necessary.

We did not define a value interpretation for type variables in lecture 3, so let us try to push on without defining that part.

The next type is $\forall \alpha. \; \tau$. When we define the value interpretation we consider the elimination forms which in this case is type application. Before we proceed let us consider one of the free theorems from lecture 3 that we wanted to be able to prove:
\begin{theorem}
  If $\mtenv \vdash \tau$, 
  $\mtenv \vdash \tau'$, 
  $\mtenv \vdash v_1 : \tau$, 
  and $\mtenv \vdash v_1 : \tau'$, 
  then $\ctxeq{e[\tau]\; v_1}{e[\tau']\; v_2} : bool$.
\end{theorem}
There are some important points to notice in this free theorem. First of all, we want to be able to apply $\Lambda$-terms to different types, so in our value interpretation we will have to pick two different types. Further, normally we pick related expressions, so it would probably be a good idea to pick related types. We do not, however, have a notion of related types, and in the theorem there is no relation between the two types used, so relating them might not be a good idea after all. With these points in mind we can make a first attempt at defining the value interpretation of $\forall \alpha. \; \tau$:
\[
  \vpred{\forall \alpha. \; \tau} = \{(\tLabs{e_1}, \tLabs{e_2}) \vbar \forall \tau_1,\tau_2. \; (\subst{e_1}{\tau_1}{\alpha},\subst{e_2}{\tau_2}{\alpha}) \in \epred{\subst{\tau}{?}{\alpha}} \}
  \]
Now the question is what type to relate the two expressions under. We need to substitute $?$ for some type, but if we use either $\tau_1$ or $\tau_2$, then the well-typedness requirement will be broken. We choose to leave $\tau$ as it is and not do the substitution. We do, however, need to keep track of what types we picked in the left and right part of the pair. To do so we use a relational substitution:
\[
  \rho = \{ \alpha_1 \mapsto (\tau_{11},\tau_{12} ), \dots \}
\]
Which we parameterize the interpretations with.
\[
  \vprep{\forall \alpha.\; \tau} = \{(\tLabs{e_1},\tLabs{e_2}) \vbar \forall \tau_1 , \tau_2. \; (\subst{e_1}{\tau_1}{\alpha},\subst{e_2}{\tau_2}{\alpha}) \in \eprep[\extsub{\rho}{\alpha}{(\tau_1,\tau_2)}]{\tau} \}
\]
We need to parameterize the entire logical relation with the relational substitution, otherwise we will not know what type to pick when we interpret the polymorphic type and we will not know how to close off the values. Which leads us to the next issue. We are now interpreting types with free type variables, so we need to have a value interpretation of type $\alpha$. It will look something like
\[
\vprep{\alpha} = \{(v_1,v_2) \vbar \rho(\alpha) = (\tau_1,\tau_2) \dots\}
\]
We need to say that the values are related, but the question is how to relate them. To figure this out we again look to the free theorem. In the free theorem, the two values are related at the argument type we choose. We therefore pick a relation on these types when we pick the types. We remember the relation we pick in the relational substitution. We finally reach our definition of the value interpretation of $\forall \alpha. \; \tau$:
\[
  \vprep{\forall \alpha.\; \tau}  = \{(\tLabs{e_1},\tLabs{e_2}) \vbar 
  \forall \tau_1 , \tau_2, R. \; 
    R \in \Rel[\tau_1,\tau_2].\;
      (\subst{e_1}{\tau_1}{\alpha},\subst{e_2}{\tau_2}{\alpha}) \in \eprep[\extsub{\rho}{\alpha}{(\tau_1,\tau_2,R)}]{\tau} \} 
\]
We do not require much of the relation, $R$. It has to be a set of pairs of values, and the values in every pair of the relation have to be closed and well typed under the corresponding type. So we define $\Rel[\tau_1,\tau_2]$ as:
\[
\Rel[\tau_1,\tau_2] = \{R \in \curly{P}(Val \times Val) \vbar \forall (v_1,v_2) \in R. \; \mtenv \vdash v_1 : \tau_1 \pand \mtenv \vdash v_2 : \tau_2 \}
\]
In the interpretation of $\alpha$ we require the values to be related under the relation we choose in the value interpretation of $\forall \alpha. \; \tau$:
\[
  \vprep{\alpha} = \{ (v_1,v_2) \vbar \rho(\alpha) = (\tau_1,\tau_2,R) \pand (v_1,v_2) \in R \}
\]
For convenience we introduce the following notation for projection in $\rho$. Given
\[
  \rho = \{\alpha_1 \mapsto (\tau_{11},\tau_{12},R_1), \alpha_2 \mapsto (\tau_{21},\tau_{22},R_2),\; \dots\; \}
\]
Define the following projections:
\begin{align*}
  \rho_1 & = \{\alpha_1 \mapsto \tau_{11}, \; \alpha_2 \mapsto \tau_{21},\; \dots \;\} \\
  \rho_2 & = \{\alpha_1 \mapsto \tau_{12}, \; \alpha_2 \mapsto \tau_{22},\; \dots \;\} \\
  \rho_R & = \{\alpha_1 \mapsto R_1, \; \alpha_2 \mapsto R_2,\; \dots \;\} 
\end{align*}
Notice that $\rho_1$ and $\rho_2$ now are type substitutions, so we write $\rho_1(\tau)$ to mean $\tau$ where all the type variables mentioned in the substitution has been substituted with the appropriate types. We can now write the value interpretation for type variables in a more succinct way:
\[
  \vprep{\alpha} = \rho_R(\alpha)
\]
%TODO: include something like: ``When we use our logical relation to prove results such as the free theorem above, then we get to choose the relation $R$ which means that we get to choose what values are related at $\alpha$!''? and ``The power of parametricity is that you pick types for $\alpha$ as well as the notion values of those types are related under.
%TODO: further: ``If we want to prove something about a term with type $\forall \alpha. \; \tau$, then we get to pick the the relation R. Whereas if we want to prove a term is of a polymorphic type, then we have the relation as a proof obligation and have to show it for an arbitrary R.
We need to add $\rho$ to the other parts of the value interpretation as well. Moreover, as we now interpret open types we require the pairs of values in the relation to be well typed under the type closed of using the relational substitution. So all value interpretations have the form
\[
  \vprep{\tau} = \{(v_1,v_2) \vbar \mtenv ; \mtenv \vdash v_1 : \rho_1(\tau) \pand \mtenv ; \mtenv \vdash v_2 : \rho_2(\tau) \pand\dots \}
\]
We further need to close of the type annotation of the variable in functions, so our value interpretations end up as:
\begin{align*}
  \vprep{bool}                 & = \{ (\true,\true), (\false,\false) \} \\
  \vprep{\tarrow{\tau}{\tau'}} & = \{ (\tlabs{x}{\rho_1(\tau)}{e_1},\tlabs{x}{\rho_2(\tau)}{e_2}) \vbar \forall (v_1,v_2) \in \vprep{\tau}. \; (\subst{e_1}{v_1}{x},\subst{e_2}{v_2}{x}) \in \eprep{\tau'} \}
\end{align*}
We define our interpretation of expressions as follows:
\begin{align*}
  \eprep{\tau} = \{(e_1,e_2) \vbar & \mtenv ; \mtenv \vdash e_1 : \rho_1(\tau) \pand \\
                                   & \mtenv ; \mtenv \vdash e_2 : \rho_2(\tau) \pand \\
                                   &  \exists v_1,v_2.\; e_1 \evaltos v_1 \pand
                                            e_2 \evaltos v_2 \pand
                                            (v_1,v_2) \in \vprep{\tau} \}
\end{align*}
We now need to give an interpretation of the contexts $\Delta$ and $\Gamma$:
\begin{align*}
  \dpred{\mtenv}         &= \{\emptyset \} \\
  \dpred{\Delta, \alpha} &= \{\extsub{\rho}{\alpha}{(\tau_1,\tau_2,R)} \vbar 
                                 \rho \in \dpred{\Delta} \pand 
                                 R \in \Rel[\tau_1,\tau_2] \} \\
  \gprep{\mtenv}         &= \{\emptyset \} \\
  \gprep{\Gamma, x:\tau} &= \{\extsub{\gamma}{x}{(v_1,v_2)} \vbar 
                                 \gamma \in \gprep{\Gamma} \pand
                                 (v_1,v_2) \in \vprep{\tau}\}
\end{align*}
We need the relational substitution in the interpretation of $\Gamma$, because $\tau$ might contain free type variables now.
%In \gpred{\Gamma, x:\tau} \tau may have free type variable in it which is why we need \rho. recall \Delta ; \Gamma \vdash e : \tau (?)
We introduce a convenient notation for the projections of $\gamma$ similar to the one we did for $\rho$: %TODO: Write something about the fact that we now want to close of pairs of expressions, so we get a substitution on pairs.
\[
  \gamma = \{\alpha_1 \mapsto (v_{11},v_{12}), \alpha_2 \mapsto (v_{21},v_{22}),\; \dots\; \}
\]
Define the projections as follows:
\begin{align*}
  \gamma_1 & = \{\alpha_1 \mapsto v_{11}, \; \alpha_2 \mapsto v_{21},\; \dots \;\} \\
  \gamma_2 & = \{\alpha_1 \mapsto v_{12}, \; \alpha_2 \mapsto v_{22},\; \dots \;\}
\end{align*}
%write \approx for \lreq{}{}
We are now ready to define when two terms are logically related. We define it in a similar way to the logical predicate we already have defined. First we pick $\rho$ and $\gamma$ to close of the terms, then we require the closed of expressions to be related under the expression interpretation of the type in question:
\renewcommand{\lreq}[2]{\equivalence{#1}{}{#2}}
\begin{align*}
  \Delta ; \Gamma \vdash \lreq{e_1}{e_2} : \tau \eqdef & \Delta ; \Gamma \vdash e_1 : \tau \pand\\
                                                      & \Delta ; \Gamma \vdash e_2 : \tau \pand \\
                                                      & \forall \rho \in \dpred{\Delta}. \; \forall \gamma \in \gprep{\Gamma}. \; (\rho_1(\gamma_1(e_1)),\rho_2(\gamma_2(e_2))) \in \eprep{\tau}
\end{align*}
Now we have defined our logical relation the first thing we want to do is to prove the fundamental property:
\begin{fundamentalprop}[Fundamental Property]
  If $\Delta; \Gamma \vdash e : \tau$ then $\Delta; \Gamma \vdash \lreq{e}{e} : \tau$
\end{fundamentalprop}
This theorem may seem a bit mundane, but it is actually quite strong. In the definition of the logical relation, $\Delta$ and $\Gamma$ can be seen as maps of place holders that needs to be replaced in the expression. So when we choose a $\rho$ and $\gamma$ we may pick different types and terms to put in the expression. Closing the expression of can then give us two very different programs. 

In some presentations this is also known as the parametricity lemma. It may even be stated with out the short-hand notation for equivalence we use here.

We could prove the theorem directly by induction over the typing derivation, but we will instead prove it by means of compatibility lemmas.
%''Parametricity'' \Delta; \Gamma <- Place holders for input. When we close of the terms we get two different programs.
%Prove directly via induction on typing judgment. One case per typing rule.
\subsubsection*{Compatibility Lemmas}
We state a compatibility for each of the typing rules we have. Each of the lemmas will correspond to a case in the induction proof of the Fundamental Property so the theorem will follow directly from the compatibility lemmas. We state the compatibility lemmas as rules to highlight the connection to the typing rules. The premises of the lemma are over the horizontal line, and the conclusion is below:
\begin{enumerate}
\item $\Gamma ; \Delta \vdash \lreq{\true}{\true} : bool$ % \approx \true ``added to typing rule'' (?)
\item $\Gamma ; \Delta \vdash \lreq{\false}{\false} : bool$
\item $\Gamma ; \Delta \vdash \lreq{x}{x} : \Gamma(x)$
\item $\inferrule*{\Delta;\Gamma \vdash \lreq{e_1}{e_2} : \tarrow{\tau'}{\tau} \and
                   \Delta;\Gamma \vdash \lreq{e_1'}{e_2'} : \tau'}
                  {\Delta;\Gamma \vdash \lreq{e_1 \; e_1'}{e_2 \; e_2'} : \tau}$
%slightly more general than we need, different things on both sides, gives us FP, but we can use them to prove more.
\item $\inferrule*{\Delta; \Gamma, x:\tau \vdash \lreq{e_1}{e_2} : \tau'}
                  {\Delta; \Gamma \vdash \lreq{\tlabs{x}{\tau}{e_1}}{\tlabs{x}{\tau}{e_2}} : \tarrow{\tau}{\tau'}}$
\item $\inferrule*{\Delta; \Gamma \vdash \lreq{e_1}{e_2} : \forall \alpha . \tau \and
                   \Delta \vdash \tau'}
                  {\Delta; \Gamma \vdash \lreq{e_1[\tau']}{e_2[\tau']} : \subst{\tau}{\tau'}{\alpha}}$
%TODO: Add if? (then delete the text that says it has been omitted
\end{enumerate}
The rule for if has been omitted here. Notice some of the lemmas are more general than what we actually need. Take for instance the compatibility lemma for expression application. To prove the fundamental property, we really just needed to have the same expressions on both sides of the equivalence. It turns out that the slightly more general version helps when we want to prove that the logical relation is sound with respect to contextual equivalence.

We will only prove the compatibility lemma for type application. To do so we are going to need the following lemma:
\begin{lemma}[Compositionality]
  Let $\Delta \vdash \tau'$, $\Delta, \alpha \vdash \tau$, $\rho \in \dpred{\Delta}$, and $R=\vprep{\tau'}$ then
\[
  \vprep{\subst{\tau}{\tau'}{\alpha}} = \vprep[\extsub{\rho}{\alpha}{(\rho_1(\tau'),\rho_2(\tau'),R)}]{\tau} 
\]
\end{lemma}
The lemma says syntactically substituting some type for $\alpha$ into $\tau$ and then interpreting it is the same as semantically substituting the type for $\alpha$. To prove this lemma we would need to show $\vprep{\tau} \in \Rel[\rho_1(\tau),\rho_2(\tau)]$ which is fairly easy given how we have defined our value interpretation.
\begin{proof}[Proof. (Compatibility, Lemma 6)]
What we want to show is 
\[
  \inferrule*{\Delta; \Gamma \vdash \lreq{e_1}{e_2} : \forall \alpha . \tau \and
              \Delta \vdash \tau'}
             {\Delta; \Gamma \vdash \lreq{e_1[\tau']}{e_2[\tau']} : \subst{\tau}{\tau'}{\alpha}}
\]
So we assume (1) $\Delta; \Gamma \vdash \lreq{e_1}{e_2} : \forall \alpha . \tau$ and (2) $\Delta \vdash \tau'$. According to our definition of the logical relation, we need to show three things:
  \begin{align*}
    & \Delta ; \Gamma \vdash e_1[\tau'] : \subst{\tau}{\tau'}{\alpha} \\
    & \Delta ; \Gamma \vdash e_2[\tau'] : \subst{\tau}{\tau'}{\alpha} \\
    & \forall \rho \in \dpred{\Delta}. \; \forall \gamma \in \gprep{\Gamma}. \; (\rho_1(\gamma_1(e_1[\tau'])),\rho_2(\gamma_2(e_2[\tau']))) \in \eprep{\subst{\tau}{\tau'}{\alpha}}
  \end{align*}
The two first follows from the well-typedness part of (1) together with (2) and the appropriate typing rule. So it only remains to show the last one.

Suppose we have a $\rho$ in $\dpred{\Delta}$ and a $\gamma$ in $\gprep{\Gamma}$. We then need to show $(\rho_1(\gamma_1(e_1[\tau'])),\rho_2(\gamma_2(e_2[\tau']))) \in \eprep{\subst{\tau}{\tau'}{\alpha}}$. From the $\curly{E}$-relation we find that to show this we need to show that the two terms run down to two values and those values are related. 

We keep this goal in mind and turn our attention to our premise, (1). This gives us by definition:
\[
\forall \rho \in \dpred{\Delta}. \; \forall \gamma \in \gprep{\Gamma}. \; (\rho_1(\gamma_1(e_1)),\rho_2(\gamma_2(e_2))) \in \eprep{\tau}
\]
If we instantiate this with the $\rho$ and $\gamma$ we supposed previously, then we get $(\rho_1(\gamma_1(e_1)),\rho_2(\gamma_2(e_2))) \in \eprep{\tau}$ which means that $e_1$ and $e_2$ runs down to some value $v_1$ and $v_2$ where $(v_1,v_2) \in \vprep{\forall \alpha. \; \tau}$. As $(v_1,v_2)$ is in the value interpretation of $\forall \alpha. \; \tau$ we know that the values are of type $\forall \alpha. \; \tau$. From this we know that $v_1$ and $v_2$ are type abstractions, so there must exist $e_1'$ and $e_2'$ such that $v_1 = \tLabs{e_1'}$ and $v_2 = \tLabs{e_2'}$. We can now instantiate $(v_1,v_2) \in \vprep{\forall \alpha. \; \tau}$ with two types and a relation. We choose $\rho_1(\tau')$ and $\rho_2(\tau')$ as the two type for the instantiation and $\vprep{\tau'}$ as the relation\footnote{Here we use $\vprep{\tau} \in \Rel[\rho_1(\tau),\rho_2(\tau)]$ to justify using the value interpretation as our relation.}. This gives us
\[
  (\subst{e_1'}{\rho_1(\tau')}{\alpha},\subst{e_2'}{\rho_2(\tau')}{\alpha}) \in \eprep[\extsub{\rho}{\alpha}{(\rho_1(\tau'),\rho_2(\tau'),\vprep{\tau'})}]{\tau}
\]
For convenience we write $\rho' = \extsub{\rho}{\alpha}{(\rho_1(\tau'),\rho_2(\tau'),\vprep{\tau'})}$. From the two expressions membership of the expression interpretation we know that $\subst{e_1'}{\rho_1(\tau)}{\alpha}$ and $\subst{e_2'}{\rho_2(\tau)}{\alpha}$ run down to some values say $v_{1_f}$ and $v_{2_f}$ respectively where $(v_{1_f},v_{2_f}) \in \vprep[\rho']{\tau}$.

Let us take a step back and see what we have done. We have argued that the following evaluation takes place
\begin{align*}
  \rho_i (\gamma_i(e_i)) [\rho_i(\tau')] & \evaltos (\tLabs{e_i'})[\rho_1(\tau')] \\
                      & \evalto \subst{e_i'}{\rho_i(\tau')}{\alpha} \\
                      & \evaltos v_{i_f}
\end{align*}
where $i=1,2$. The single step in the middle is justified by the type application reduction. The remaining steps are justified in our proof above. If we further note that $\rho_i (\gamma_i (e_i [\tau'])) \equiv  \rho_i (\gamma_i (e_i))[\rho_1(\tau')]$, then we have shown that the two expressions from our goal in fact do run down to two values, and they are related. More precisely we have:
\[
  (v_{1_f},v_{2_f}) \in \vprep[\rho']{\tau}
\]
but that is not exactly what we wanted them to be related under. We are, however, in luck and can apply the compositionality lemma to obtain
\[
  (v_{1_f},v_{2_f}) \in \vprep{\subst{\tau}{\tau'}{\alpha}}
\]
which means that they are related under the relation we needed.
\end{proof}

We call theorems that follows as a consequence of parametricity for free theorems. Next, we will show a free theorem that says that an expression of the type $\forall \alpha. \; \tarrow{\alpha}{\alpha}$ must be the identity function.

\begin{theorem}[Free Theorem (I)] If $\mtenv; \mtenv \vdash e : \forall \alpha.\; \tarrow{\alpha}{\alpha}$, $\mtenv \vdash \tau$, and $\mtenv; \mtenv \vdash v : \tau$, then
\[
  e[\tau] \; v \evaltos v
\]
\end{theorem}
System-F is a terminating language, so in the free theorem it suffices to say that when it terminates, then it is with the value passed as argument. If we had been in a non-terminating language such as System-F with recursive types, then we would have had to state a weaker theorem namely: the expression terminates with the value given as argument as result, or the computation diverges.
\begin{proof}
\newcommand{\aaa}{\ensuremath{\forall \alpha. \; \tarrow{\alpha}{\alpha}}}
From the fundamental property and the well-typedness of $e$ we know $\mtenv \vdash \lreq{e}{e} : \aaa$. By definition this gives us
\[
\forall \rho \in \dpred{\Delta}. \; \forall \gamma \in \gprep{\Gamma}. \; (\rho_1(\gamma_1(e)),\rho_2(\gamma_2(e))) \in \eprep{\aaa}
\]
\newcommand{\mt}{\ensuremath{\emptyset}}
We instantiate this with an empty $\rho$ and an empty $\gamma$ to get $(e,e) \in \eprep[\mt]{\aaa}$. From the definition of this we know that $e$ evaluates to some value $F$ and $(F,F) \in \vprep[\mt]{\aaa}$. As $F$ is a value of type \aaa{} we know $F=\tLabs{e_1}$ for some $e_1$. Now use the fact that $(F,F) \in \vprep[\mt]{\aaa}$ by instantiating it with the type $\tau$ twice and the relation $R=\{(v,v)\}$ to get 
\newcommand{\env}{\ensuremath{\extsub{\mt}{\alpha}{(\tau,\tau,R)}}}
\newcommand{\taa}{\ensuremath{\tarrow{\alpha}{\alpha}}}
$(\subst{e_1}{\tau}{\alpha},\subst{e_1}{\tau}{\alpha})\in \eprep[\env]{\taa}$. We notice that this instantiation is all right as $R \in \Rel[\tau,\tau]$.

This step is an important part of a proof of any free theorem namely choosing the relation. Before we chose the relation we picked two types. We did this based on the theorem we want to show. In the theorem we instantiate $e$ with $\tau$, so we pick $\tau$. Likewise with the relation, in the theorem we give $v$ to the function with the domain $\alpha$, so we pick the singleton relation consisting of $(v,v)$. Picking the correct relation is what requires some work in the proof of a free theorem. The remaining work done in the proof is simply unfolding of definitions.

Now let us return to the proof. From $(\subst{e_1}{\tau}{\alpha},\subst{e_1}{\tau}{\alpha})\in \eprep[\env]{\taa}$ we know that $\subst{e_1}{\tau}{\alpha}$ evaluates to some value $g$ and $(g,g)\in \vprep[\env]{\taa}$. From the type of $g$ we know that it must be a $\lambda$-abstraction, so $g=\tlabs{x}{\tau}{e_2}$ for some expression $e_2$. Now instantiate $(g,g)\in \vprep[\env]{\taa}$ with $(v,v) \in \vprep[\env]{\alpha}$ to get $(\subst{e_2}{v}{x},\subst{e_2}{v}{x}) \in \eprep[\env]{\alpha}$. From this we know that $\subst{e_2}{v}{x}$ steps to some value $v_f$ and $(v_f,v_f) \in \vprep[\env]{\alpha}$. We have that $\vprep[\env]{\alpha} \equiv R$ so $(v_f,v_f) \in R$ which mean that $v_f = v$ as $(v,v)$ is the only pair in $R$.

Now let us take a step back and consider what we have shown above.
\begin{align*}
  e[\tau] \; v & \evaltos F [\tau] \; v \\
               & \equiv (\tLabs{e_1}) [\tau] \; v \\
               & \evalto (\subst{e_1}{\tau}{\alpha}) \; v \\
               & \evaltos g \; v\\
               & \equiv (\tlabs{x}{\tau}{e_2}) \; v \\
               & \evalto \subst{e_2}{v}{x} \\
               & \evaltos v_f \\
               & \equiv v
\end{align*}
First we argued that $e[\tau]$ steps to some $F$ and that $F$ was a type abstraction, \tLabs{e_1}. Then we performed the type application to get $\subst{e_1}{\tau}{\alpha}$. We then argued that this steps to some $g$ of the form $\tlabs{x}{\tau}{e_2}$ which further allowed us to do a $\beta$-reduction to obtain $\evalto \subst{e_2}{v}{x}$. We then argued that this reduced to $v_f$ which was the same as $v$. In summation we argued $e[\tau] \; v \evaltos v$ which is the result we wanted.
\end{proof}
%We can take a number of steps and end up with v.
%choice of R was critical.
%\forall \alpha. \alpha easier to show (?)
\subsection*{Exercises}
\begin{enumerate}
\item Prove the following free theorem:
  \begin{theorem}[Free Theorem (II)]
    If $\mtenv; \mtenv \vdash e : \forall \alpha. \; (\tarrow{(\tarrow{\tau}{\alpha})}{\alpha})$ and 
       $\mtenv; \mtenv \vdash k : \tarrow{\tau}{\tau_k}$ then
    \[
      \mtenv; \mtenv \vdash \lreq{e[\tau_k] \; k}{k(e[\tau] \; \tlabs{x}{\tau}{x})} : \tau_k
%''\mtenv; \mtenv \vdash'' not in my notes
    \]
  \end{theorem}
  This theorem is a simplified version of the one found in \emph{Theorems For Free} by Philip Wadler. %TODO: insert proper reference (http://ttic.uchicago.edu/~dreyer/course/papers/wadler.pdf)
\end{enumerate}

